\documentclass[journal,onecolumn]{IEEEtran}
\IEEEoverridecommandlockouts
% The preceding line is only needed to identify funding in the first footnote. If that is unneeded, please comment it out.
\usepackage{cite}
\usepackage{amsmath,amssymb,amsfonts}
\usepackage{algorithmic}
\usepackage{graphicx}
\usepackage{textcomp}
\usepackage{xcolor}
\usepackage{hyperref}
\usepackage{listings}
\usepackage{color} %red, green, blue, yellow, cyan, magenta, black, white
% http://www.mathworks.com/matlabcentral/fileexchange/8015-m-code-latex-package
\usepackage[framed,numbered,autolinebreaks,useliterate]{mcode/mcode}


\renewcommand\refname{REFERENCIAS}
\def\BibTeX{{\rm B\kern-.05em{\sc i\kern-.025em b}\kern-.08em
    T\kern-.1667em\lower.7ex\hbox{E}\kern-.125emX}}
\begin{document}

\title{Implementación del cifrado Cesar en Matlab}

\author{
    García García Jonathan Eduardo\\
    Instituto Politécnico Nacional\\
    Escuela Superior de Ingeniería Mecánica y Eléctrica Unidad Culhuacán\\
    Maestría en Ingeniería en Seguridad y Tecnologías de la información 
    Ciudad de México, México\\
    \href{jgarciag1404@alumno.ipn.mxm}{jgarciag1404@alumno.ipn.mx} 
     }
\maketitle

\section{Introducción}
\noindent
La función esencial de la criptografía es mantener la privacidad de la
comunicación, de forma que el mensaje sea inteligible tan solo para sus destinatarios.
Esta necesidad de ocultar cierta información ha estado siempre presente en la vida del
ser humano, datando el primer sistema criptográfico del que se tiene constancia del siglo
V a.C. 

La palabra “criptografía” proviene
etimológicamente del griego Kriptos (ocultar), Graphos
(escritura), “ocultar la escritura”. En un sentido más
amplio significa aplicar alguna técnica para hacer
ininteligible un mensaje. La criptografía es una
herramienta muy útil cuando se desea tener seguridad
informática; puede ser también entendida como un
medio para garantizar las propiedades de
confidencialidad, integridad y disponibilidad de los
recursos de un sistema. \\
La criptologia consiste
en permitir el intercambio de información a través de un
medio de comunicación inseguro, de forma que, si la
información es interceptada por un intruso, sea
imposible su descifrado\\
Esta ciencia está dividida en dos
grandes ramas:
\begin{itemize}
    \item La criptografía: ocupada del cifrado de
    mensajes en clave y del diseño de criptosistemas
    \item El criptoanálisis: que trata de descifrar los
    mensajes en clave, rompiendo así el criptosistema.
\end{itemize}

\section{Cifrado Cesar}
\noindent
El cifrado de César es un cifrado de textos planos muy
antiguo, se cree que Julio César lo utilizó para dirigir
mensajes confidenciales a sus generales en campañas
militares en el primer siglo antes de Cristo. Se basa en
un cifrado mono alfabético.\\
En criptografía, un cifrado César se clasifica como
un cifrado por sustitución en el que el alfabeto en el
texto plano se desplaza por un número fijo en el alfabeto
\subsection{Ventajas}
\begin{itemize}
    \item Uno de los métodos más fáciles de usar en
    criptografía y puede proporcionar una
    seguridad mínima a la información.
    \item Uno de los mejores métodos para usar
    si el sistema no puede usar ninguna técnica de
    codificación complicada
\end{itemize}

\subsection{Desventajas}
\begin{itemize}
    \item Uso de estructura simple
    \item Solo puede proporcionar seguridad mínima a la
    información
    \item La frecuencia del patrón de letras proporciona
    una gran pista para descifrar el mensaje
    completo.
\end{itemize}

\section{Análisis de frecuencia}
En que consiste el análisis de frecuencia, en cualquier
idioma tenemos unas letras más comunes que otras.
\begin{enumerate}
    \item Se analiza la frecuencia de cada letra del idioma
    español (inglés o portugués) y a cada letra se le da
    un valor determinado en función de su frecuencia
    de uso.
    \item Se analizan diferentes textos en español (inglés o
    portugués) para validar los valores anteriores 
    \item Una vez obtenida esta tabla se lee
    el texto cifrado  
    \item Se reconoce el texto cifrado y se hace un análisis
    de la frecuencia de aparición de los diferentes
    símbolos. Se crea una nueva tabla con estos
    resultados.
    \item Los valores de ambas tablas se comparan luego por
    su frecuencia y los símbolos del texto cifrado se
    sustituyen por las letras del alfabeto
    correspondientes.
\end{enumerate}

\section{Desarrollo}
Para este trabajo, se requirió programar una
aplicación que pueda mostrar el resultado de un texto
plano tras ser cifrado con el algoritmo de Cesar.\\
Implementación función para cifrado y descifrado de una cadena de texto
\lstinputlisting{code/cipher.m}
\newpage
Para el uso de este cifrado se implemento una interfaz grafica que cuenta con las siguientes funcionalidades:
\begin{enumerate}
    \item Selección de archivo de texto
    \item Selección de idioma
    \item Selección de llave de cifrado
    \item Campo de texto para ingresar texto a tratar
    \item Campo de texto donde se muestra el resultado del texto
    \item Opciones para cifrar,Limpiar y descifrar el texto
\end{enumerate}

\begin{figure}[!ht]
    \centering
    \includegraphics[width=1\textwidth]{imgs/window1.jpg}
    \caption{GUI para el cifrado}
    \label{fig_sim}
    \end{figure}


\section{Conclusiones}
El método de cifrado cesar es una de las formas mas simples que tenemos de cifrar un texto y sorprende la antigüedad del mismo, en tiempos donde el análisis de frecuencias no era una realidad debido a los alcances de los medios impresos este cifrado resultó medianamente efectivo pues recordemos que aún es susceptible a ataques de fuerza bruta o de ingeniería social para conocer la llave de desplazamiento
Asimismo, se deberían plantear cuestiones que resulten, al menos, interesantes para
realizar como trabajos futuros.
\begin{thebibliography}{00}
\bibitem{b1}Gómez Hernández, S. (2010). Análisis de textos cifrados de los siglos XVI y XVII (Bachelor's thesis).
\bibitem{b2} Paredes, G. G. (2006). Introducción a la Criptografía.
\end{thebibliography} 

\end{document}
