\documentclass[conference]{IEEEtran}
\IEEEoverridecommandlockouts
% The preceding line is only needed to identify funding in the first footnote. If that is unneeded, please comment it out.
\usepackage{cite}
\usepackage{amsmath,amssymb,amsfonts}
\usepackage{algorithmic}
\usepackage{graphicx}
\usepackage{textcomp}
\usepackage{xcolor}
\usepackage{hyperref}
\renewcommand\refname{REFERENCIAS}
\def\BibTeX{{\rm B\kern-.05em{\sc i\kern-.025em b}\kern-.08em
    T\kern-.1667em\lower.7ex\hbox{E}\kern-.125emX}}
\begin{document}

\title{Controles ciberseg IC\\

\thanks{Identify applicable funding agency here. If none, delete this.}
}

\author{\IEEEauthorblockN{García García Jonthan Eduardo}
\IEEEauthorblockA{
\href{mailto:jgarciag1404@alumno.ipn.mx}{jgarciag1404@alumno.ipn.mx}
}
\and
\IEEEauthorblockN{González Santiesteban Santiago}
\IEEEauthorblockA{
    \href{mailto:sgonzalezs1400@alumno.ipn.mx}{sgonzalezs1400@alumno.ipn.mx}
}
\and
\IEEEauthorblockN{Jiménez Angeles Daniel Antonio}
\IEEEauthorblockA{
    \href{mailto:djimeneza1400@alumno.ipn.mx}{djimeneza1400@alumno.ipn.mx}
}
}

\maketitle

\section{Resumen}
La infraestructura de las redes de telecomunicaciones tienen una gran relevancia para el desarrollo y crecimiento del país; el transporte de los datos y la interconexión hacia las diversas redes de la que dependen sectores que brindan servicios esenciales para el país deben cobrar la importancia y permear en la conciencia de los actores sobre su importancia estratégica así como darse los recursos necesario para garantizar su operación ante amenazas y vulnerabilidades a la que está expuesta la infraestructura así como el costo asociado a los riesgos no atendidos que puede llegar a tener un impacto mucho mayor a la implementación de protección requerida. 

En ese sentido las telecomunicaciones deben considerarse un tema de infraestructura crítica y con ello,  un plan de continuidad de negocio para cada uno de los actores puede garantizar en su conjunto la operación de la cual depdenden diversos sectores escenciales.

\section{Introducción}
El  Plan  de  Continuidad  del  Negocio,  tiene  como  objetivo  principal  proteger  los  procesos  críticos  del  negocio,   contra   desastres,   sean   estos   naturales,   humanos  o  tecnológicos,  y  evaluar  las    posibles  secuelas  que  se  puedan  tener,  como  pérdidas  de  tipo  financiero, operativo, productividad, etc., debido a la no disponibilidad de los recursos de la compañía. \\ El  Plan  de  Continuidad  del  Negocio,  es  esencial  para  la  continuidad  de    las  actividades  críticas  del  negocio,  en  el  caso  de  que  se  presentara  un  evento    inesperado  que  pudiera  comprometer  los  procesos  y  actividades   importantes   de   la   operación   de   la   compañía.

\section{Fases de continuidad del plan de negocio.}
El Plan de continuidad del negocio, se conforma de un conjunto de directrices y
procedimientos plasmados en un documento técnico, para que cada entidad pueda
tomar las acciones pertinentes con miras a la recuperación y restablecimiento de
los servicios e infraestructuras de Telecomunicaciones interrumpidas por situaciones de desastre o emergencias ocurridas en cualquier instante dentro de las organizaciones.

El análisis de impacto del negocio como parte del plan de continuidad del negocio,
debe entenderse como un marco conceptual sobre el cual las entidades deben
planear integralmente los alcances y objetivos, que permiten proteger la
información, en todas sus áreas críticas.
Las entidades deben establecer un análisis de impacto del negocio, que este
alineado con el Plan General de Continuidad del Negocio de la Entidad; este debe
tener una estrategia de continuidad de TI, que contenga los objetivos globales de la
entidad, con respecto a las dimensiones de disponibilidad de datos, infraestructura
tecnológica y recurso humano.

Para desarrollar el plan de continuidad del negocio de TI se debe tener en cuenta:

\begin{itemize}
    \item Diseñar una estrategia de continuidad de los servicios de Telecomunicaciones, que tenga como base la reducción del impacto de una interrupción en los servicios
críticos del negocio, este debe estar difundido, aprobado y respaldado por los directivos de la entidad.
    \item Realizar un análisis e identificación de recursos críticos de comunicaciones así como de TI vitales, de esta manera se establece una estrategia que genere prioridades en caso de
presentarse una o varias situaciones que causen interrupciones.
    \item Establecer procedimientos de control de cambio, que permita asegurar que
el plan de continuidad, se encuentre actualizado y permita afrontar las
amenazas que traen consigo las nuevas tendencias tecnológicas sin perder
el alcance de los requerimientos de la Entidad.
    \item Elaborar un plan de pruebas de continuidad de TI, que permita verificar y
asegurar que los sistemas de TI, puedan ser recuperados de forma segura y efectiva, atendiendo y corrigiendo errores, que atenten contra la disponibilidad de las operaciones.
    \item Realizar capacitaciones del plan de continuidad de Telecomunicaciones y análisis de impacto del negocio, a los entes o partes involucradas de la organización (Equipo de
seguridad de sistemas de información de la entidad), para que conozcan cuáles son sus roles y responsabilidades en caso de incidentes o desastres. Es necesario verificar e incrementar el entrenamiento de acuerdo con los resultados de las pruebas de contingencia generadas
dentro de la entidad.
    \item Tanto el plan de continuidad de TI como el análisis de impacto del negocio
deben estar disponibles apropiadamente dentro de la organización y en manos de los responsables de las áreas de TI quienes de forma segura deben garantizar su aplicabilidad en los momentos críticos, a su vez la entidad debe propender por un plan de sensibilización al interior de la
misma con el propósito de indicar a todos sus miembros sobre la importancia de contar con un plan de continuidad y de análisis del negocio que van a garantizar el normal funcionamiento de las operaciones regulares en caso de presentarse problemas críticos en los sistemas de información y comunicaciones de la entidad.
\end{itemize}

\subsection{Metodología del análisis para el impacto del negocio.}

La metodología del Análisis de Impacto del Negocio, consiste en definir una serie
de pasos interactivos con el objeto de identificar claramente los impactos de las
interrupciones y tomar decisiones respecto a aquellos procesos que se consideran
críticos para la organización y que afectan directamente el negocio ante la
ocurrencia de un desastre, estos pasos se muestran en esta ilustración:

\begin{figure}[htbp]
    \begin{center}
    \includegraphics[scale=.37]{image/Imagen5.png}
    \caption{Indice de impacto}
\label{fig0}
    \end{center}
\end{figure} 

\subsection{Evaluación de impactos operacionales.}

Teniendo en cuenta los elementos operacionales de la organización, se requiere
evaluar el nivel de impacto de una interrupción dentro de la Entidad.
El impacto operacional permite evaluar el nivel negativo de una interrupción en
varios aspectos de las operaciones del negocio; el impacto se puede medir
utilizando un esquema de valoración, con los siguientes niveles: A, B o C.

\begin{itemize}
    \item \textbf{Nivel A:} La operación es crítica para el negocio. Una operación es crítica cuando al no contar con ésta, la función del negocio no puede realizarse.
    \item \textbf{Nivel B:} La operación es una parte integral del negocio, sin ésta el negocio no podría operar normalmente, pero la función no es crítica.
    \item \textbf{Nivel C:} La operación no es una parte integral del negocio.
\end{itemize}

\subsection{Marco teórico}
\textbf{RTO (Recovery Time Objective)}
\\
Es el máximo tiempo   permitido   que   un   proceso   puede   estar   detenido seguido a un evento desastroso. 
\\
\textbf{RPO  (Recovery  Point  Objective)}\\
Identificar  si  para la recuperación del proceso que se haya visto afectado  se  necesita  disponer  de  la  información  que   se   tenía   justo   antes   de   que   sucediera   el   incidente, o si, por el contrario, se puede utilizar la información anterior.

\subsection{Descripción de la empresa}

\subsection{Misión de la empresa}

\subsection{Visión de la empresa}

\section{Evaluar los impactos en cinco escenarios de servicio dentro de la infraestructura de Telecomunicaciones}
\subsection{Interrupciones de la cobertura}
El impacto de las interrupciones de cobertura de la telecomunicación provoca que haya problemas en los distintos servicios que lo ocupan, por ejemplo en el siguiente mapa se muestra los estados que se verían afectados por la interrupción de cobertura afectando así a personas que pudieran tener alguna emergencia y necesitan del uso del celular para comunicarse así como algunas aplicaciones bancarias que se necesita el factor de autenticación mandando un SMS para un código de acceso dificultando su uso por lapsos de tiempo aquellas redes de telecomunicaciones, cuya interrupción o destrucción podría producir un serio impacto en la salud, seguridad o bienestar de la población o producir un serio impacto en el funcionamiento del gobierno o de la economía del país.
Durante la crisis que vivimos actualmente a causa del conocido Coronavirus, las telecomunicaciones tienen un rol importante en diferentes ámbitos diarios.
En el plano social, es obvio que cambian la manera las personas de afrontar la cuarentena impuesta tras el Estado de alarma.
\begin{figure}[htbp]
    \begin{center}
    \includegraphics[width=0.4\textwidth]{image/Imagen1.png}
    \caption{Cobertura Altán}
\label{fig1}
    \end{center}
\end{figure} 

\subsection{Mal servicio al clientes}\label{AA}
El impacto del mal servicio al cliente de la telecomunicación provoca que haya problemas varios usuarios decidan buscar más opciones de telefonía provocando que la empresa tenga disminución de ingresos lo cual indirectamente lleva al despido de personas ya que no son necesarias por la disminución de usuarios y generando multas a la empresa por el mal servicio proporcionado a la población, se observa que las llamadas telefónicas se intensificaron desde la pandemia del COVID-19 y que, además, se vuelve imperioso el poder gestionar estas llamadas de los equipos de venta desde las redes móviles, lo que permitirá a los equipos seguir con el teletrabajo o, incluso, gestionar estas llamadas desde cualquier otro sitio fuera de la oficina y un mal funcionamiento del servicio afectaria de manera crucial estos puntos.

\subsection{Perdida de servicio}
El impacto de la perdida de servicio en telecomunicación provoca que haya problemas en los distintos servicios que lo ocupan, esto sería un impacto serio ya que al no tener servicio estaría afectado a servicios externos como aplicaciones bancarias y al usuario en general ya que en casos de vida o muerte la falta de servicio seria un punto crítico para el resultado de este que al no
tenerla disponible, puede ocasionar pérdidas de vida, serio o grave impacto en la salud, seguridad o economía de sus ciudadanos la extensión geográfica que se afecta en caso de un siniestro.

\begin{figure}[htbp]
    \begin{center}
    \includegraphics[width=0.3\textwidth]{image/Imagen2.png}
    \caption{Indice de impacto}
\label{fig1}
    \end{center}
\end{figure} 
En la perdida de servicio de Altan afectaría los siguientes sectores principalmente:
\begin{figure}[htbp]
    \begin{center}
    \includegraphics[width=0.5\textwidth]{image/Imagen3.png}
    \caption{Cobertura Altán}
\label{fig1}
    \end{center}
\end{figure} 
\\
El sector de las telecomunicaciones no es únicamente partícipe de los retos sociales y económicos, sino que se postula como un actor clave ya que se volvío un punto importante en la actual pandemia de Covid19 ya que aumentaron las demandas de servicio por parte de los usuarios y la perdiada del mismo servicio seria un impacto critico para esos momentos criticos ya que al no tener posibilidad de una llamada o mensaje por falta de servicio seria de vida o muerte.
\subsection{Quiebra de la empresa}

El impacto de la quiebra de una empresa de telecomunicación provoca que haya problemas en muchos sectores de la población ya que muchas empresas contratan planes para sus empleados además de esto muchas aplicaciones necesitan el número telefónico del usuario para autenticación, así como el hecho de que también se les afecta por el hecho de buscar otra compañía que les ofrezca los beneficios que la actual compañía daba y que posiblemente no se ajusten a sus posibilidades y/o ingresos.\\
\begin{figure}[htbp]
    \begin{center}
    \includegraphics[width=0.3\textwidth]{image/Imagen4.png}
    \caption{Quiebra}
\label{fig1}
    \end{center}
\end{figure} 
\\
La noticia fue dada a conocer señalando que la empresa pidió rediseñar el plan y prorrogar la fecha a 2028. Esta solicitud debe ser aprobada por el gobierno federal, el IFT y publicarse en el Diario Oficial de la Federación (DOF). Hasta el momento, la empresa alcanzó una cobertura de red del 61,1 por ciento, equivalente a 68 millones de habitantes, mientras que el restante 30,71 por ciento, es decir, 34 millones de personas serán tocadas recién en siete años.
\subsection{Pandemia(Covid19)}\label{SCM}
El COVID-19 también trajo consigo una aceleración del paso de los Call Center a la nube. Con la pandemia del COVID-19 quedó claro que la telefonía fija, ante estos casos extraordinarios y ante las restricciones sanitarias impuestas, no funcionarían del todo debido a la imposibilidad de acudir a las oficinas. Si los Call Center deseaban estar preparados, deberían mover su infraestructura de negocio a la nube para así trabajar fácilmente desde cualquier lugar, con cualquier dispositivo conectado a Internet y que esto no reduzca su productividad.  
Los centros de atención al cliente tuvieron que adaptarse a la rápida evolución de las normas de confinamiento en casa y otras restricciones que afectaron a los lugares de trabajo y a las plantillas de todo el mundo,las aplicaciones que están basadas en la infraestructura de telecomunicaciones amortiguo una de las principales características que impone el COVID el aislamiento.
El impacto fue grande ya que la industria tuvo que adaptarse y evolucionar de manera acelerada par cumplir con las necesidades de la sociedad con este percanse y un impacto mayor hubiera provocado el no contar con las comunicaciones ya que muchas personas no habrian tenido la posibilidad del teletrabajo, escuela en linea y demas necesidades que habrian tenido un gran impacto en la poblacion.
\section{Estandarés}
\section{NIST Cybersecurity Framework}
El NIST (Instituto Nacional de Estándares y Tecnología) de los
Estados Unidos, posee un “Marco para la mejora de la Seguridad cibernética
en infraestructuras críticas”, actualmente se encuentra disponible en su
versión 1.1 publicado el 16 de Abril del 2018, el que ofrece, la posibilidad de
implementar mejoras en lo que refiere a la Ciberseguridad, teniendo en
cuenta, que existen riesgos específicos para cada tipo de Organización
acorde a su función dentro de la estructura del Estado en sí, lo que se busca
con la implementación es reducir y gestionar mejor los riesgos de la
seguridad cibernética.\\
Visto lo mencionado anteriormente, alineado con la problemática en el
que se encontraban las Organizaciones Gubernamentales de los Estados
Unidos de Norteamérica, en lo concerniente a la ciberseguridad ya que eran
blancos constantes de los ciber atacantes, lo que demostraba la necesidad
de una mejora continua en esta especificidad, es que nace el Marco de
Trabajo de Ciberseguridad del NIST (National Institute of Standards and
Technology), en el período de Barack Obama como presidente, a través de
la Orden Ejecutiva Nro. 13.636 del 12 de febrero del 2013, la cual consta de
12 secciones, siendo el foco principal la protección de las Infraestructuras
Críticas de ese País, introduciéndolas dentro del ámbito de aplicación de la
Seguridad Nacional y la Economía; ya que son la base de lo antes
mencionado, definiéndolas como todo sistema y activos, ya sean físicos o
virtuales, tan vitales que la incapacidad o destrucción de tales sistemas y
activos tendría un impacto debilitante en la seguridad, la seguridad
económica nacional, la opinión pública nacional. Salud o seguridad, o
cualquier combinación de esos asuntos
\subsection{Marco de trabajo}
Las bases del Marco de Trabajo se pueden desprender directamente
de la Orden Ejecutiva Nro. 13.636, siendo las siguientes:
\begin{itemize}
    \item El centro del Marco es el conjunto de funciones que engloban la
seguridad cibernética siendo comunes en todos los sectores de la
Organizaciones.
\item Definir una guía para el manejo de los riesgos de ciberseguridad
como parte de la gestión de los procesos de la Organización.
\item Enfocar en la aplicación de iniciadores de negocios para la totalidad
de las actividades desarrolladas para la ciberseguridad.
\item Ayudará a la Organización a alinear y priorizar los recursos y/o
actividades de seguridad cibernética, teniendo en cuenta el negocio.
\item El marco representa un documento dinámico, en constante mejora, en
base a las recomendaciones y comentarios de las Organizaciones
que lo implementen.
\item Evitar la implementación de nuevos estándares, si ya se cuenta con
una normativa que ya cubre o abarque los descritos en dicha orden
ejecutiva.
\end{itemize}

\section{Controles de recuperación (RTO,RPO y MTD)}

\begin{table}[h!]
\begin{tabular}{|l|l|l|l|}
\hline
Vulnerabilidad   - Amenaza          & Probabilidad & Impacto & Valoración \\ \hline
Terremotos & BAJO         & ALTO & BAJO\\ \hline
Inundaciones & BAJO & ALTO & BAJO\\ \hline
Tornados & BAJO & ALTO & BAJO\\ \hline
Avalanchas & BAJO & ALTO & BAJO\\ \hline
Tormentas eléctricas & BAJO & BAJO & BAJO\\ \hline
Deslizamientos de   tierra & BAJO & ALTO & BAJO\\ \hline
Error humano   & BAJO & BAJO    & BAJO\\ \hline
Ataques a la red & BAJO & ALTO & BAJO\\ \hline
Incendio & BAJO & ALTO & BAJO\\ \hline
Fallas en el   antivirus & BAJO & MEDIO & BAJO\\ \hline
Acceso no autorizado & BAJO & MEDIO   & ALTO\\ \hline
Fallas del servicio eléctrico & BAJO & ALTO & BAJO\\ \hline
Polución/humedad & BAJO & MEDIO & BAJO\\ \hline
Químicos & BAJO & MEDIO & BAJO\\ \hline
Robo de información & BAJO & MEDIO & BAJO\\ \hline
Daño a las copias de   respaldo & BAJO & MEDIO & BAJO\\ \hline
Fallas en las comunicaciones & MEDIO & ALTO & ALTO \\ \hline
Caida de servidor especifico & MEDIO & MEDIO & MEDIO \\ \hline
Daño en fisicos de discos duros & BAJO & BAJO & BAJO \\ \hline
Falla en el los aplicativos & BAJO & MEDIO & MEDIO \\ \hline
Espionaje corporativo & BAJO & ALTO & MEDIO \\ \hline
Falta de experiencia & BAJO & MEDIO & BAJO \\ \hline
Falta de mantenimiento & BAJO & MEDIO & BAJO \\ \hline
Información confidencial & BAJO & ALTO & BAJO \\ \hline
Falla trasmisión de datos & MEDIO & MEDIO & MEDIO \\ \hline
Hurto & BAJO & ALTO & BAJO \\ \hline
Falla estado de los cables & BAJO & ALTO & BAJO \\ \hline
Falla en la red de datos & MEDIO & ALTO & ALTO \\ \hline
Fallas en la red de voz & BAJO & ALTO & ALTO \\ \hline

\end{tabular}
\caption{Tiempos de recuperación de los servicios críticos frente a un evento de vulnerabilidad.}
\end{table}

\begin{table}[h!]
\begin{tabular}{|l|l|}
\hline
\textbf{Valoración} & RT en horas \\ \hline
BAJO                & 72          \\ \hline
MEDIO               & 12-24       \\ \hline
ALTO                & Inmediato   \\ \hline
\end{tabular}
\end{table}

\begin{itemize}
    \item \textbf{Fallas del servicio eléctrico}\\
    Las temporadas invernales incrementan la probabilidad de fallos en el suministro eléctrico, y aun cuando hay contingencia y la disponibilidad del servicio casi siempre se preserva, hay casos en los que puede fallar o tardar tiempo en entrar la misma
    \item \textbf{Falla en la trasmisión de datos}
    Debido al incremento de
aplicativo que consumen el
canal de comunicación de la
empresa, se ha notado
momentos en los que
aplicativos consumen recursos
y debilitan las trasmisiones de
datos corriendo el riesgo de
que algun aplicativo consuma
la totalidad de los recursos
    \item \textbf{Fallas en las comunicaciones}\\
    Debido a que la
comunicación entre sedes
esta subcontratada con
terceros, estos tienen
problemas sobre todo con las
sedes A y B, en la primera
porque el contratista sufre
constantemente el robo de las
fibras de comunicación y el
segundo porque los
constantes cortes de energía
eléctrica en el municipio
de Tubará, descontrolan los
enlaces de comunicación
suministrados por el proveedor
\end{itemize}


En muchos casos, algunos de las vulnerabilidades identificados en el análisis
de riesgos y análisis de impacto se pueden mitigar con controles preventivos.
Estos controles son apropiados para reducir costos en la gestión de riesgos
ya que reducen el impacto y la probabilidad fácilmente. 
Entre ellos tenemos:
\begin{itemize}
    \item Plantas generadoras de energía a base de gasolina que proveen por
un buen tiempo.
\item Sistemas de aire refrigerado para mantener controlada la
temperatura del sitio a proteger.
\item Sistemas contra incendios.
\item Censores de agua y humedad en paredes y piso
\item Almacenamiento de externo de copias de respaldo
\item Apropiadas fuentes de poder eléctrico que proveen energía
suficiente en caso de emergencia.
\item Sistemas de aire refrigerado para mantener controlada la
temperatura del sitio a proteger
\item Detectores de humo y fuego.
\end{itemize}

Las razones por la cuales se debe escoger muy la estrategia que
recuperación es porque será esta quien devolverá a la normalidad los
diferentes recursos que se ven afectados en la eventualidad de un
desastre.
\\
Los factores determinantes son: el costo, los datos a proteger, tipos de
controles ya instalados, requerimientos adicionales.
Como se puede verificar así en el documento de NIST SP 800 - 34:
\begin{itemize}
    \item Los sitios en frío son normalmente las instalaciones físicas con un
espacio adecuado y la infraestructura (energía eléctrica, las
conexiones de telecomunicaciones, y los controles del medio
ambiente) para apoyar las actividades de recuperación del sistema de
información.
\item Los sitios en tibio están parcialmente equipados con espacios de
oficinas que contienen algunos o todos los hardware del sistema, el
software, las telecomunicaciones y las fuentes de energía necesarias
para la recuperación.
\item Sitios en caliente son centros de tamaño adecuado para soportar los
requerimientos del sistema y configurado con el hardware del sistema
necesario y actualizado, la infraestructura de apoyo, y personal de
apoyo”
\end{itemize}

\subsection{Recovery Time Objective (RTO)}
El RTO, Recovery Time Objective o Tiempo Objetivo de Recuperación, se refiere al tiempo máximo que una empresa define para recuperar sus procesos críticos, después de haber sido afectada por alguna una contingencia.
\\
Si ocurre algo que impida a la empresa operar de manera normal y afecte sus operaciones, habrá un tiempo en el que no podrán operar hasta volver a reanudar todos sus procesos. El RTO es ese tiempo desde el momento de la interrupción, hasta el reinicio de las actividades.

\subsection{Recovery Point Objective (RPO)}
Se relaciona con la copia de seguridad de los datos dentro del mismo escenario de un eventual desastre que afecte al negocio. Es el tiempo máximo que se establece desde la última copia de seguridad relacionado a la cantidad de datos que el negocio puede permitirse perder en caso de desastre.
\\
Cualquiera sea el servicio que ofrezcas a tus clientes, en caso de desastres este servicio y sus datos pueden perderse. Dentro del Plan de Continuidad del negocio y el de Recuperación ante Desastres, la empresa necesita saber cuánto tiempo puede permitirse el no disponer del servicio y la cantidad de información que se puede asumir como perdida, antes de recuperar la Continuidad Operacional.
\\
Con esto en mente, se definen las réplicas o copias de seguridad. Y el RPO será la cantidad de datos que el negocio puede permitirse perder, para seguir funcionando.

\subsection{Maximum Tolerable Downtime (MTD)}
Tiempo máximo tolerable de caída el cual nos determina el tiempo que puede estar caído un proceso antes de que se produzcan efectos desastrosos en la compañía y repercuta en el negocio.

\subsection{Principales beneficios obtenidos de desarrollar un análisis de impacto sobre el negocio}
\begin{itemize}
    \item Se delimitan los procesos o actividades críticas dentro de la organización que afectan a nuestro negocio pudiendo descubrir actividades críticas que a priori no lo parecían.
    \item Permite identificar vulnerabilidades de una organización en materia de continuidad de negocio.
    \item En caso de disponer de planes de recuperación permitirá verificar si estos cubren las necesidades del negocio.
    \item Propicia la implicación de un mayor número de áreas de la organización a la hora de implantar planes de continuidad, no solo al personal responsable de llevar a término este tipo de proyectos.
    \item Reducción de costes ante posibles interrupciones del negocio.
    \item Aporta información de gran valor a la hora de priorizar el desarrollo de otros proyectos en materia de continuidad de negocio.
    \item Un mayor conocimiento de los procesos de negocio, contribuirá favorablemente a la mejora de la competitividad y seguridad en el mercado.
    \item La información obtenida en el desarrollo del BIA es una base fundamental para implantar estrategias de recuperación eficientes.
\end{itemize}


\begin{table*}[h]
\begin{tabular}{|l|l|l|l|c|c|}
\hline
\textbf{Recurso}   & \textbf{Descripción}                                                                                                                                   & \textbf{Requerimientos}                                                                                                                     & Responsable                                                               & RTO - Horas & RPO -Horas \\ \hline
Telecomunicaciones & \begin{tabular}[c]{@{}l@{}}Es   la plataforma de comunicaciones \\ con la cual todos los sistemas se \\ interrelacionan automatizadamente\end{tabular} & \begin{tabular}[c]{@{}l@{}}Dispositivos Activos, redes, \\ servicios de proveedores,\\  instalaciones físicas, \\ electricidad\end{tabular} & \begin{tabular}[c]{@{}l@{}}Jefe de \\ Infraestructura\end{tabular}        & 3           & 0          \\ \hline
Bases de datos     & \begin{tabular}[c]{@{}l@{}}Fuente de datos para todas las aplicaciones\\  operacionales\end{tabular}                                                   & \begin{tabular}[c]{@{}l@{}}Servidores, conexiones, \\ clientes, plataforma, \\ electricidad\end{tabular}                                    & DBA                                                                       & 3           & 0          \\ \hline
Aplicaciones       & \begin{tabular}[c]{@{}l@{}}Programas de ofimática, y actividades \\ de soporte\end{tabular}                                                            & \begin{tabular}[c]{@{}l@{}}Computadores, electricidad,\\  plataforma, conexiones\end{tabular}                                               & Jefe de soporte                                                           & 3           & 0          \\ \hline
Ejecutables ERP    & \begin{tabular}[c]{@{}l@{}}Equipo informático que administra el \\ servicio de ERP\end{tabular}                                                        & \begin{tabular}[c]{@{}l@{}}PC, Electricidad, \\ Conexión\end{tabular}                                                                       & \begin{tabular}[c]{@{}l@{}}Jefe de Plataforma\\  tecnológica\end{tabular} & 0,75        & 0          \\ \hline
Terminal Server    & \begin{tabular}[c]{@{}l@{}}Equipo informático que administra el \\ servicio de terminal Server\end{tabular}                                            & \begin{tabular}[c]{@{}l@{}}PC, Electricidad,\\  Conexión\end{tabular}                                                                       & \begin{tabular}[c]{@{}l@{}}Jefe de Plataforma\\  tecnológica\end{tabular} & 0.75        & 2          \\ \hline
Ejecutables ERP    & \begin{tabular}[c]{@{}l@{}}Equipo informático que administra el \\ servicio de ERP\end{tabular}                                                        & \begin{tabular}[c]{@{}l@{}}PC, Electricidad, \\ Conexión\end{tabular}                                                                       & \begin{tabular}[c]{@{}l@{}}Jefe de Plataforma \\ tecnológica\end{tabular} & 0.75        & 0          \\ \hline
\end{tabular}
\end{table*}

\begin{figure}[h!]
\centerline{\includegraphics[width=0.4\textwidth]{image/RPO.png}}
\caption{Grafica de RPO con Recursos de Ti críticos.}
\label{fig}
\end{figure}

\begin{figure}[h!]
\centerline{\includegraphics[width=0.4\textwidth]{image/RTO.png}}
\caption{Grafica de RPO con Recursos de Ti críticos.}
\label{fig}
\end{figure}
\newpage

Para  tomar  una  decisión  acertada  y  razonable,  
en  la  que  dos  criterios  solicitados  por  gerencia  
tengan distintas opiniones es preferible y 
recomendable  la  intervención  de  un  tercero  que  
sea experto en el tema para de esta manera 
solucionar el problema y llegar a un acuerdo. 
Es recomendable que se realice frecuentemente 
un  Análisis  de  Riesgos  del  Negocio  en  todas  sus  
áreas, con el objetivo de mantenerse siempre alerta 
ante cualquier circunstancia e implementar un 
Plan  de  Continuidad  del  Negocio  como  medida  
para estos casos.  

\newpage
\begin{thebibliography}{00}
\bibitem{b1}Rusi, T. (2019, July). Applicability of Resilience Metrics in the Context of Telecommunications Services. In European Conference on Cyber Warfare and Security (pp. 845-XVIII). Academic Conferences International Limited.
\bibitem{b2} Saadawi, T., \& Chamas, H. (2015). Securing telecommunications infrastructure against cyber attacks. Geo. J. Int'l Aff., 16, 58.
\bibitem{b3} Shen, L. (2014). The NIST cybersecurity framework: Overview and potential impacts. Scitech Lawyer, 10(4), 16.
\bibitem{b4}https://historicoupress.upaep.mx/index.php/opinion/editoriales/innovacion-y-tecnologia/6679-las-telecomunicaciones-en-tiempos-de-pandemia
\bibitem{b5}https://www.comunycarse.com/es/la-importancia-de-las-redes-moviles-en-los-equipos-de-ventas-tras-el-covid-19/
\bibitem{b6}https://medux.com/es/covid-19-que-supone-para-la-industria-de-las-telecomunicaciones/
\bibitem{b7}https://www.altanredes.com/en/solutions-to-operators/our-coverage/
\end{thebibliography} 

\end{document}
