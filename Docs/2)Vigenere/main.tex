\documentclass[10pt]{article}

%%%%%%%% PREÁMBULO %%%%%%%%%%%%
\usepackage[spanish,es-tabla]{babel}
\usepackage[utf8]{inputenc} 
\usepackage{multirow} % para las tablas
\usepackage{makecell}
\usepackage{tcolorbox}
\tcbuselibrary{listingsutf8}
\usepackage[spanish]{babel}
\usepackage{fancybox}
\usepackage{courier}
\usepackage{ragged2e}
\usepackage[spanish]{babel}
\usepackage[utf8]{inputenc}
\usepackage{amssymb, amsmath, amsbsy} % simbolitos
\usepackage{upgreek} % para poner letras griegas sin cursiva
\usepackage{cancel} % para tachar
\usepackage{mathdots} % para el comando \iddots
\usepackage{mathrsfs} % para formato de letra
\usepackage{stackrel} % para el comando \stackbin
\usepackage{courier}
\usepackage{subfig}
\usepackage{pdflscape}
\title{Plantilla Tesis SEPI ESIME}
\usepackage[spanish]{babel} %Indica que escribiermos en español
\usepackage[utf8]{inputenc} %Indica qué codificación se está usando ISO-8859-1(latin1)  o utf8  
\usepackage{amsmath} % Comandos extras para matemáticas (cajas para ecuaciones, etc)
\usepackage{amssymb} % Simbolos matematicos (por lo tanto)
\usepackage{graphicx} % Incluir imágenes en LaTeX
\graphicspath{{./../imgs/}}
\usepackage{color} % Para colorear texto
%\usepackage{subfigure} % subfiguras
\usepackage{float} %Podemos usar el especificador [H] en las figuras para que se queden donde queramos
\usepackage{capt-of} % Permite usar etiquetas fuera de elementos flotantes (etiquetas de figuras)
\usepackage{sidecap} % Para poner el texto de las imágenes al lado
\sidecaptionvpos{figure}{c} % Para que el texto se alinie al centro vertical
\usepackage{caption} % Para poder quitar numeracion de figuras
\usepackage{commath} % funcionalidades extras para diferenciales, integrales, etc (\od, \dif, etc)
\usepackage{cancel} % para cancelar expresiones (\cancelto{0}{x})
\usepackage[export]{adjustbox}
\usepackage{anysize} % Para personalizar el ancho de  los márgenes
\marginsize{2cm}{2cm}{2cm}{2cm} % Izquierda, derecha, arriba, abajo

\usepackage{appendix} 
\renewcommand{\appendixname}{Apéndices}
\renewcommand{\appendixtocname}{Apéndices}
\renewcommand{\appendixpagename}{Apéndices} 

% Para que las referencias sean hipervínculos a las figuras o ecuaciones y aparezcan en color
\usepackage[colorlinks=true,plainpages=true,citecolor=blue,linkcolor=blue]{hyperref}
%\usepackage{hyperref}
%Para agregar encabezado y pie de página
\usepackage{fancyhdr} 
\pagestyle{fancy}
\fancyhf{}
\fancyhead[L]{\footnotesize ESIME Culhuacan} %encabezado izquierda
\fancyhead[R]{\footnotesize IPN}   % dereecha
\fancyfoot[R]{\footnotesize E.S.I.M.E }  % Pie derecha
\fancyfoot[C]{\thepage}  % centro
\fancyfoot[L]{\footnotesize MISTI}  %izquierda
\renewcommand{\footrulewidth}{0.4pt}

\usepackage{listings} % Para usar código fuente
\definecolor{dkgreen}{rgb}{0,0.6,0} % Definimos colores para usar en el código
\definecolor{gray}{rgb}{0.5,0.5,0.5} 

% configuración para el lenguaje que queramos utilizar
\usepackage[framed,numbered,autolinebreaks,useliterate]{./../mcode/mcode}

\newcommand{\sen}{\operatorname{\sen}}	% Definimos el comando \sen para el seno en español


\usepackage{longtable}
%%%%%%%% TERMINA PREÁMBULO %%%%%%%%%%%%

\begin{document}
%\nocite{IEEEreferencias:Ref1}
\begin{center}											            							%%%
    \newcommand{\HRule}{\rule{\linewidth}{0.5mm}}	                                               %%%\left
    %%%
    \noindent
\makebox[0pt][l]{
    \begin{minipage}{0.49\textwidth} \begin{flushleft}
        \includegraphics[scale = 0.12]{logoesime.png}
        \end{flushleft}\end{minipage}
}\hfill\makebox[0pt]
{
    \begin{minipage}{0.49\textwidth} \begin{center}
        \includegraphics[scale = 0.15]{misti.png}
        \end{center}\end{minipage}
}\hfill%
\makebox[0pt][r]{
    \begin{minipage}{0.49\textwidth} \begin{flushright}
        \includegraphics[scale = 0.25]{IPN.png}
        \end{flushright}\end{minipage}
}\\
\vspace*{1cm}			%%%
    
    \textsc{\huge Instituto Polit\'ecnico Nacional}\\[1cm] 
    \textsc{\LARGE Escuela Superior de Ingenier\'ia Mecanica y Electrica}\\[0.5cm] %%%
    \textsc{\LARGE Unidad Culhuacan}\\[0.5cm] %%%
    \textsc{\LARGE Maestría en Ingeniería en Seguridad y Tecnologías de la información }\\[1cm] %%%
    
    \begin{minipage}{0.9\textwidth} 
    \begin{center}																					%%%
    \textsc{\LARGE REPORTE TÉCNICO}
    \end{center}
    \end{minipage}\\[0.5cm]
                                                                                        %%%
                                                                                        %%%
                 \vspace*{1cm}															%%%
                                                                                        %%%
    \HRule \\[0.1cm]																	%%%
    \begin{center} \textsc{\Large Implementación del Cifrado de Vigenére en Matlab\\}
    \end{center}
    \HRule \\[0.1cm]%%%
                                             %%%
    
                                                                                         %%%
    
    
    
       \vspace{0.8cm}
    \begin{center}
    {\large Presenta}\\                                                                %%%
    Jonathan Eduardo García García\hspace{1cm} \href{mailto:jgarciag1404@alumno.ipn.mx}{jgarciag1404@alumno.ipn.mx}
    \vspace{1 cm}
    \end{center}
    
    \begin{center}
    \begin{minipage}{1\textwidth}													    %%%
    \begin{flushleft} \large														    %%%
    \emph{Profesor:}\\
    \vspace{0.3cm}
    Dr. José Portillo Portillo\\
        Introducción a los sistemas de comunicación seguros.
        \vspace*{1cm}	
        %%%
    \end{flushleft}	
    %%%
    \end{minipage}		                                            %%%
    \end{center}
    
    %\begin{minipage}{1\textwidth}													    %%%
    %\begin{flushleft} \large														    %%%
    %\emph{Directores del proyecto:}\\
    %\vspace{0.3cm}
    %M. en C. Jose Luis Cano Rosas\\
    %Dr. Pedro Guevara L\'opez\
    
                                                                    %%%
     %   \vspace*{1cm}	
                                                                    %%%
    %\end{flushleft}													%%%
    %\end{minipage}		                                            %%%
                                                                    %%%
    %\begin{flushleft}
         
    %\end{flushleft}
    %%%
             
                         %%%
    \vspace{2cm} 																				
    \begin{center}
    {\large \today}													%%%
    \end{center}												  						
    
    \end{center}
\cleardoublepage


\newpage																		
\tableofcontents 





%%%%%%%%%%%%%%%%%%%%%%%%%%%%%%%%%%%%%%%%%%%%%%%%%%%%%%%%%%%%%%%%%%%%%%%%%%%%%%%%%%%%%%%
%RESUMEN
\newpage
\section{Objetivo.}

\begin{enumerate}
  \item Implementar el cifrado de Vigenère en matlab
  \item Crear una interfaz de usuario para hacer uso del cifrado
  \item Generar los histogramas del texto cifrado, texto plano y del idioma
\end{enumerate}

  \begin{center}
    \begin{tabular}{ | l | l |}
      \hline
      \thead{\textbf{Equipo necesario}} & \thead{\textbf{Material necesario}}  \\
      \hline
      \makecell[l]{Computadora con el Software Matlab.}&  
			\makecell[l]{Apuntes y conocimientos teóricos sobre el cifrado de Vigenère}  \\
      \hline
    \end{tabular}
  \end{center}

\section{Marco teórico}\index{Introducción o Marco teórico}
\subsection{Cifrado de Vigenére}
\justify
El cifrado de Vigenère está basado en el cifrado del Cesar, por lo cual es un cifrado de sustitución. A diferencia 
del cifrado del cesar, en el cual cada símbolo del texto 
plano le es sumada una constante k, en el cifrado de 
Vigenère se tiene un cifrado del Cesar por cada símbolo de 
una palabra clave. Con lo cual si la palabra clave tiene una 
longitud m, se tienen m corrimientos diferentes sobre el 
texto encriptado. De esta forma, no siempre un mismo 
símbolo en el texto claro se convierte en el mismo símbolo 
en el texto encriptado.\\
A cada letra del texto plano se le 
sumaría una letra de la clave y como la clave suele ser de 
menor longitud que el texto plano se repetiría para lograr el 
tamaño del texto plano.\\
Formalmente el cifrado de Vigenère se puede expresar de la 
siguiente manera.\\
$C_i=S_i+K_{i mod(m)}mod(n)$

\subsection{Entropia}
\justify
La entropía es un concepto valioso cuando se piensa en hacer 
criptoanálisis dado que representa la medida promedio de 
información que tiene un símbolo en algún mensaje, de hecho se 
puede pensar en calcular la entropía para cierto lenguaje 
(español, inglés, etc.) y es curioso saber que la entropía de cada 
lenguaje tiende a cierto valor característico.
La cantidad de información de un símbolo B se define como:\\
$I(B)=\log_2{\frac{1}{P(B)}}$

\section{Desarrollo}
\justify

Se tienen los siguientes alfabetos con su respectiva frecuencia:\\
\textbf{Español:}
\begin{table}[h]
    \begin{adjustbox}{width=\columnwidth,center}
    \begin{tabular}{|l|l|l|l|l|l|l|l|l|l|l|l|l|l|l|l|l|l|l|l|l|l|l|l|l|l|l|}
    \hline
    A & B & C & D & E & F & G & H & I & J  & K  & L  & M  & N  & Ñ  & O  & P  & Q  & R  & S  & T  & U  & V  & W  & X  & Y  & Z  \\ \hline
    1 & 2 & 3 & 4 & 5 & 6 & 7 & 8 & 9 & 10 & 11 & 12 & 13 & 14 & 15 & 16 & 17 & 18 & 19 & 20 & 21 & 22 & 23 & 24 & 25 & 26 & 27\\ \hline
    \%12.53&\%1.42&\%4.68&\%5.86&\%13.68&\%0.69&\%1.01&\%0.70&\%6.25&\%0.44&\%0.02&\%4.97&\%3.15&\%6.71&\%0.31&\%8.68&\%2.51&\%0.88&\%6.87&\%7.98&\%4.63&\%3.93&\%0.90&\%0.01&\%0.22&\%0.90&\%0.52 \\ \hline
    \end{tabular}
\end{adjustbox}
    \end{table}


    \textbf{Inglés:}
    \begin{table}[h]
        \begin{adjustbox}{width=\columnwidth,center}
        \begin{tabular}{|l|l|l|l|l|l|l|l|l|l|l|l|l|l|l|l|l|l|l|l|l|l|l|l|l|l|l|}
        \hline
        A & B & C & D & E & F & G & H & I & J  & K  & L  & M  & N  & O  & P  & Q  & R  & S  & T  & U  & V  & W  & X  & Y  & Z  \\ \hline
        1 & 2 & 3 & 4 & 5 & 6 & 7 & 8 & 9 & 10 & 11 & 12 & 13 & 14 & 15 & 16 & 17 & 18 & 19 & 20 & 21 & 22 & 23 & 24 & 25 & 26 \\ \hline
        \%8.34&\%1.54&\%2.73&\%4.14&\%12.6&\%2.03&\%1.92&\%6.11&\%6.71&\%0.23&\%0.87&\%4.24&\%2.53&\%6.80&\%7.70&\%1.66&\%0.09&\%5.68&\%6.11&\%9.37&\%2.85&\%1.06&\%2.34&\%0.20&\%2.04&\%0.06 \\ \hline
        \end{tabular}
    \end{adjustbox}
        \end{table}

\lstinputlisting{./../code/cipher.m}

\section{Resultados}
\justify
a

\par\vspace{\baselineskip}
%%%%%%%%%%%%%%%%%%%%%%%%%%%%%%%%%%%%%%%%%%%%%%%%%%%%%%%%%%%%%%%%%%%%%%%%%%%%%%%%%%%%%%%
\newpage

\section{Conclusión}
\justify
El presente proyecto de inicio a fin implicó retos importantes para cada uno de los miembros del equipo, de ellos podemos destacar el aprendizaje para obtener acceso en las implementaciones de las vistas de widget, notificaciones y otras funcionalidades nativas en cada plataforma así como en la homologación de la experiencia de usuario. Esta aplicación resuelve las necesidades que fueron identificadas durante la etapa de investigación del desarrollo. Durante la etapa de pruebas y puesta a disposición de los pilotos se observó una aceptación positiva por parte de los alumnos y su retroalimentación confirma que esta aplicación es de utilidad en la organización diaria de sus actividades escolares. Es importante destacar el apoyo que se tuvo por parte de las materias cursadas en la carrera que dieron las bases en el diseño y construcción de la aplicación.

%%%%%%% Bibliografía %%%%%%%%
\bibliographystyle{bst/IEEEtran.bst} 
\addcontentsline{toc}{section}{Referencias}  
\bibliography{bib/IEEEabrv,bib/IEEEreferencias.bib} 
%%%%%%% Bibliografía %%%%%%%%   

\section{Anexos}
\begin{itemize}
    \item \href{https://drive.google.com/file/d/15KZvm6CIDkM9g0DKMy8wb9b2fiIv2ROG/view?usp=sharing}
    {Diagrama arquitectura,  15 jun 2021}
    \item \href{https://drive.google.com/file/d/1LhLRHHiWVeEFRqgwnUttkpgct9kEACTq/view?usp=sharing} {Diseño Pantalla de tareas,  16 jun 2021}
    \item \href{https://drive.google.com/file/d/1N4PZQ41kn7JNZdWo4EaobB4sipUvPd8S/view?usp=sharing}{Recordatorios, 8 jul 2021}
    \item \href{https://drive.google.com/file/d/1NR3xwahN6aiRnEtGttwPc7ubX188eRmh/view?usp=sharing}{Modelo vista controlador,  6 abril 2021}
    \item \href{https://drive.google.com/file/d/1OMvjYk0Nwr1IXoy3Gtce-1zTW5y6Ea2j/view?usp=sharing}{Notas por materia, 11 jul 2021}
    \item \href{https://drive.google.com/file/d/1UC7XmlBGHwrKTog4K-RgmHMUh44nNHE9/view?usp=sharing}{Login, 25 jun 2021}
    \item \href{https://drive.google.com/file/d/1dKLuDtAksGwQyNf9vmHAt3SzZTOF-dzH/view?usp=sharing}{Pantalla información escolar, 17 ago 2021}
    \item \href{https://drive.google.com/file/d/1f6cP91edCqel2Af6U4ORgZ19fTHvJkgj/view?usp=sharing}{Esquema de notificaciones,  5 jul 2021}
    \item \href{https://drive.google.com/file/d/1gRQ1DOsUV1Sh1PLnGl-oD76absCRSu9o/view?usp=sharing}
{Esquema de servidor,  20 may}
    \item \href{https://drive.google.com/file/d/1iKIqTDIi4M3aPXLC7cX7fYwQdsycTmLQ/view?usp=sharing}
{Pantalla principal,  11 feb 2021}
    \item \href{https://drive.google.com/file/d/1tSR0xFik_Uz0kGiwxsTpW4rQPAiCZbmC/view?usp=sharing}
    {LinksPage,  1 ago 2021}
    \item \href{https://drive.google.com/file/d/1wgIdhQ5l7WvCdJyJS8HcCM8C_K1TBXVj/view?usp=sharing}{Archivero Virtual,  7 abr 2021}
    \item \href{https://drive.google.com/file/d/1ymBzTNposUKI1SH6fAeV3bMEkRAF2EMk/view?usp=sharing}{Tareas,  23 mar 2021}
    

\par\vspace{\baselineskip}

\end{itemize}

\end{document}