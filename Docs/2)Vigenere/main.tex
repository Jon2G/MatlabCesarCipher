\documentclass[10pt]{article}

%%%%%%%% PREÁMBULO %%%%%%%%%%%%
\usepackage[spanish,es-tabla]{babel}
\usepackage[utf8]{inputenc} 
\usepackage{multirow} % para las tablas
\usepackage{makecell}
\usepackage{tcolorbox}
\tcbuselibrary{listingsutf8}
\usepackage[spanish]{babel}
\usepackage{fancybox}
\usepackage{courier}
\usepackage{ragged2e}
\usepackage[spanish]{babel}
\usepackage[utf8]{inputenc}
\usepackage{amssymb, amsmath, amsbsy} % simbolitos
\usepackage{upgreek} % para poner letras griegas sin cursiva
\usepackage{cancel} % para tachar
\usepackage{mathdots} % para el comando \iddots
\usepackage{mathrsfs} % para formato de letra
\usepackage{stackrel} % para el comando \stackbin
\usepackage{courier}
\usepackage{subfig}
\usepackage{pdflscape}
\title{Plantilla Tesis SEPI ESIME}
\usepackage[spanish]{babel} %Indica que escribiermos en español
\usepackage[utf8]{inputenc} %Indica qué codificación se está usando ISO-8859-1(latin1)  o utf8  
\usepackage{amsmath} % Comandos extras para matemáticas (cajas para ecuaciones, etc)
\usepackage{amssymb} % Simbolos matematicos (por lo tanto)
\usepackage{graphicx} % Incluir imágenes en LaTeX
\graphicspath{{./../imgs/}}
\usepackage{color} % Para colorear texto
%\usepackage{subfigure} % subfiguras
\usepackage{float} %Podemos usar el especificador [H] en las figuras para que se queden donde queramos
\usepackage{capt-of} % Permite usar etiquetas fuera de elementos flotantes (etiquetas de figuras)
\usepackage{sidecap} % Para poner el texto de las imágenes al lado
\sidecaptionvpos{figure}{c} % Para que el texto se alinie al centro vertical
\usepackage{caption} % Para poder quitar numeracion de figuras
\usepackage{commath} % funcionalidades extras para diferenciales, integrales, etc (\od, \dif, etc)
\usepackage{cancel} % para cancelar expresiones (\cancelto{0}{x})
\usepackage[export]{adjustbox}
\usepackage{anysize} % Para personalizar el ancho de  los márgenes
\marginsize{2cm}{2cm}{2cm}{2cm} % Izquierda, derecha, arriba, abajo

\usepackage{appendix} 
\renewcommand{\appendixname}{Apéndices}
\renewcommand{\appendixtocname}{Apéndices}
\renewcommand{\appendixpagename}{Apéndices} 

% Para que las referencias sean hipervínculos a las figuras o ecuaciones y aparezcan en color
\usepackage[colorlinks=true,plainpages=true,citecolor=blue,linkcolor=blue]{hyperref}
%\usepackage{hyperref}
%Para agregar encabezado y pie de página
\usepackage{fancyhdr} 
\pagestyle{fancy}
\fancyhf{}
\fancyhead[L]{\footnotesize ESIME Culhuacan} %encabezado izquierda
\fancyhead[R]{\footnotesize IPN}   % dereecha
\fancyfoot[R]{\footnotesize E.S.I.M.E }  % Pie derecha
\fancyfoot[C]{\thepage}  % centro
\fancyfoot[L]{\footnotesize MISTI}  %izquierda
\renewcommand{\footrulewidth}{0.4pt}

\usepackage{listings} % Para usar código fuente
\definecolor{dkgreen}{rgb}{0,0.6,0} % Definimos colores para usar en el código
\definecolor{gray}{rgb}{0.5,0.5,0.5} 

% configuración para el lenguaje que queramos utilizar
\usepackage[framed,numbered,autolinebreaks,useliterate]{./../mcode/mcode}

\newcommand{\sen}{\operatorname{\sen}}	% Definimos el comando \sen para el seno en español


\usepackage{longtable}
%%%%%%%% TERMINA PREÁMBULO %%%%%%%%%%%%

\begin{document}
\nocite{IEEEreferencias:Ref1}
\nocite{IEEEreferencias:Ref2}
\nocite{IEEEreferencias:Ref3}
\nocite{IEEEreferencias:Ref4}
\nocite{IEEEreferencias:Ref5}


\begin{center}											            							%%%
    \newcommand{\HRule}{\rule{\linewidth}{0.5mm}}	                                               %%%\left
    %%%
    \noindent
\makebox[0pt][l]{
    \begin{minipage}{0.49\textwidth} \begin{flushleft}
        \includegraphics[scale = 0.12]{logoesime.png}
        \end{flushleft}\end{minipage}
}\hfill\makebox[0pt]
{
    \begin{minipage}{0.49\textwidth} \begin{center}
        \includegraphics[scale = 0.15]{misti.png}
        \end{center}\end{minipage}
}\hfill%
\makebox[0pt][r]{
    \begin{minipage}{0.49\textwidth} \begin{flushright}
        \includegraphics[scale = 0.25]{IPN.png}
        \end{flushright}\end{minipage}
}\\
\vspace*{1cm}			%%%
    
    \textsc{\huge Instituto Polit\'ecnico Nacional}\\[1cm] 
    \textsc{\LARGE Escuela Superior de Ingenier\'ia Mecanica y Electrica}\\[0.5cm] %%%
    \textsc{\LARGE Unidad Culhuacan}\\[0.5cm] %%%
    \textsc{\LARGE Maestría en Ingeniería en Seguridad y Tecnologías de la información }\\[1cm] %%%
    
    \begin{minipage}{0.9\textwidth} 
    \begin{center}																					%%%
    \textsc{\LARGE REPORTE TÉCNICO}
    \end{center}
    \end{minipage}\\[0.5cm]
                                                                                        %%%
                                                                                        %%%
                 \vspace*{1cm}															%%%
                                                                                        %%%
    \HRule \\[0.1cm]																	%%%
    \begin{center} \textsc{\Large Implementación del Cifrado de Vigenére en Matlab\\}
    \end{center}
    \HRule \\[0.1cm]%%%
                                             %%%
    
                                                                                         %%%
    
    
    
       \vspace{0.8cm}
    \begin{center}
    {\large Presenta}\\                                                                %%%
    Jonathan Eduardo García García\hspace{1cm} \href{mailto:jgarciag1404@alumno.ipn.mx}{jgarciag1404@alumno.ipn.mx}
    \vspace{1 cm}
    \end{center}
    
    \begin{center}
    \begin{minipage}{1\textwidth}													    %%%
    \begin{flushleft} \large														    %%%
    \emph{Profesor:}\\
    \vspace{0.3cm}
    Dr. José Portillo Portillo\\
        Introducción a los sistemas de comunicación seguros.
        \vspace*{1cm}	
        %%%
    \end{flushleft}	
    %%%
    \end{minipage}		                                            %%%
    \end{center}
    
    %\begin{minipage}{1\textwidth}													    %%%
    %\begin{flushleft} \large														    %%%
    %\emph{Directores del proyecto:}\\
    %\vspace{0.3cm}
    %M. en C. Jose Luis Cano Rosas\\
    %Dr. Pedro Guevara L\'opez\
    
                                                                    %%%
     %   \vspace*{1cm}	
                                                                    %%%
    %\end{flushleft}													%%%
    %\end{minipage}		                                            %%%
                                                                    %%%
    %\begin{flushleft}
         
    %\end{flushleft}
    %%%
             
                         %%%
    \vspace{2cm} 																				
    \begin{center}
    {\large \today}													%%%
    \end{center}												  						
    
    \end{center}
\cleardoublepage


\newpage																		
\tableofcontents 





%%%%%%%%%%%%%%%%%%%%%%%%%%%%%%%%%%%%%%%%%%%%%%%%%%%%%%%%%%%%%%%%%%%%%%%%%%%%%%%%%%%%%%%
%RESUMEN
\newpage
\section{Objetivo.}

\begin{enumerate}
  \item Implementar el cifrado de Vigenère en matlab
  \item Crear una interfaz de usuario para hacer uso del cifrado
  \item Generar los histogramas del texto cifrado, texto plano y del idioma
\end{enumerate}

  \begin{center}
    \begin{tabular}{ | l | l |}
      \hline
      \thead{\textbf{Equipo necesario}} & \thead{\textbf{Material necesario}}  \\
      \hline
      \makecell[l]{Computadora con el Software Matlab.}&  
			\makecell[l]{Apuntes y conocimientos teóricos sobre el cifrado de Vigenère}  \\
      \hline
    \end{tabular}
  \end{center}

\section{Marco teórico}\index{Introducción o Marco teórico}
\subsection{Cifrado de Vigenére}
\justify
El cifrado de Vigenère está basado en el cifrado del Cesar, por lo cual es un cifrado de sustitución. A diferencia 
del cifrado del cesar, en el cual cada símbolo del texto 
plano le es sumada una constante k, en el cifrado de 
Vigenère se tiene un cifrado del Cesar por cada símbolo de 
una palabra clave. Con lo cual si la palabra clave tiene una 
longitud m, se tienen m corrimientos diferentes sobre el 
texto encriptado. De esta forma, no siempre un mismo 
símbolo en el texto claro se convierte en el mismo símbolo 
en el texto encriptado.\\
A cada letra del texto plano se le 
sumaría una letra de la clave y como la clave suele ser de 
menor longitud que el texto plano se repetiría para lograr el 
tamaño del texto plano.\\
Formalmente el cifrado de Vigenère se puede expresar de la 
siguiente manera.\\
$C_i=S_i+K_{i mod(m)}mod(n)$

\subsection{Entropia}
\justify
La entropía es un concepto valioso cuando se piensa en hacer 
criptoanálisis dado que representa la medida promedio de 
información que tiene un símbolo en algún mensaje, de hecho se 
puede pensar en calcular la entropía para cierto lenguaje 
(español, inglés, etc.) y es curioso saber que la entropía de cada 
lenguaje tiende a cierto valor característico.
La cantidad de información de un símbolo B se define como:\\
$I(B)=\log_2{\frac{1}{P(B)}}$

\section{Desarrollo}
\justify

Se tienen los siguientes alfabetos con su respectiva frecuencia:\\
\textbf{Español:}
\begin{table}[h]
    \begin{adjustbox}{width=\columnwidth,center}
    \begin{tabular}{|l|l|l|l|l|l|l|l|l|l|l|l|l|l|l|l|l|l|l|l|l|l|l|l|l|l|l|}
    \hline
    A & B & C & D & E & F & G & H & I & J  & K  & L  & M  & N  & Ñ  & O  & P  & Q  & R  & S  & T  & U  & V  & W  & X  & Y  & Z  \\ \hline
    1 & 2 & 3 & 4 & 5 & 6 & 7 & 8 & 9 & 10 & 11 & 12 & 13 & 14 & 15 & 16 & 17 & 18 & 19 & 20 & 21 & 22 & 23 & 24 & 25 & 26 & 27\\ \hline
    \%12.53&\%1.42&\%4.68&\%5.86&\%13.68&\%0.69&\%1.01&\%0.70&\%6.25&\%0.44&\%0.02&\%4.97&\%3.15&\%6.71&\%0.31&\%8.68&\%2.51&\%0.88&\%6.87&\%7.98&\%4.63&\%3.93&\%0.90&\%0.01&\%0.22&\%0.90&\%0.52 \\ \hline
    \end{tabular}
\end{adjustbox}
    \end{table}

    \begin{figure}[!ht]
      \centering
      \includegraphics[width=0.5\textwidth]{histogramaSpanish.png}
      \caption{Histograma de frecuencias del idioma Español}
      \label{fig_sim}
    \end{figure}

    \textbf{Inglés:}
    \begin{table}[h]
        \begin{adjustbox}{width=\columnwidth,center}
        \begin{tabular}{|l|l|l|l|l|l|l|l|l|l|l|l|l|l|l|l|l|l|l|l|l|l|l|l|l|l|l|}
        \hline
        A & B & C & D & E & F & G & H & I & J  & K  & L  & M  & N  & O  & P  & Q  & R  & S  & T  & U  & V  & W  & X  & Y  & Z  \\ \hline
        1 & 2 & 3 & 4 & 5 & 6 & 7 & 8 & 9 & 10 & 11 & 12 & 13 & 14 & 15 & 16 & 17 & 18 & 19 & 20 & 21 & 22 & 23 & 24 & 25 & 26 \\ \hline
        \%8.34&\%1.54&\%2.73&\%4.14&\%12.6&\%2.03&\%1.92&\%6.11&\%6.71&\%0.23&\%0.87&\%4.24&\%2.53&\%6.80&\%7.70&\%1.66&\%0.09&\%5.68&\%6.11&\%9.37&\%2.85&\%1.06&\%2.34&\%0.20&\%2.04&\%0.06 \\ \hline
        \end{tabular}
    \end{adjustbox}
        \end{table}
        \begin{figure}[!ht]
          \centering
          \includegraphics[width=0.5\textwidth]{histogramaSpanish.png}
          \caption{Histograma de frecuencias del idioma Inglés}
          \label{fig_sim}
        \end{figure}
  \subsection{Proceso para obtener la llave de cifrado como vector numerico }
  \lstinputlisting{./../code/keyVigenere.m}

    Se toman los caracteres de la llave y se busca su representación numérica en el alfabeto:
    \begin{table}[h!]
      \begin{adjustbox}{width=0.3\columnwidth,center}
        \begin{tabular}{|l|l|l|l|l|l|}
      \hline
      Llave & L  & L  & A & V  & E \\ \hline
      Valor & 12 & 12 & 1 & 22 & 5 \\ \hline
    \end{tabular}
  \end{adjustbox}
\end{table}
\\
Estos valores servirán como desplazamiento para nuestro texto original repitiendo el patrón hasta terminar la cadena que queremos cifrar.
\\A cada letra del texto plano se le 
sumaría una letra de la clave y como la clave suele ser de 
menor longitud que el texto plano se repetiría para lograr el 
tamaño del texto plano.
\newpage


\subsection{Implementación del cifrado en matlab}      
\lstinputlisting{./../code/vigenere.m}

  Para cifrar y descifrar el texto plano se siguen los siguientes pasos:
  \begin{enumerate}
    \item Se obtiene la llave de cifrado en su representación numérica 
    \begin{table}[h!]
      \begin{adjustbox}{width=0.3\columnwidth,center}
        \begin{tabular}{|l|l|l|l|l|l|}
      \hline
      Llave & L  & L  & A & V  & E \\ \hline
      Valor & 12 & 12 & 1 & 22 & 5 \\ \hline
    \end{tabular}
  \end{adjustbox}
\end{table}
    \item Por cada letra en el texto plano se obtiene su representación numérica
    \begin{table}[h]
      \begin{adjustbox}{width=1\columnwidth,center}
      \begin{tabular}{|l|l|l|l|l|l|l|l|l|l|l|l|l|l|l|l|l|l|l|l|l|l|l|l|l|l|l|}
      \hline
      Texto   & A  & B  & C & D  & E  & F  & G  & H & I  & J  & K  & L  & M  & N  & O  & P  & Q  & R  & S  & T  & U  & V  & W  & X  & Y  & Z  \\ \hline
      Valor   & 1  & 2  & 3 & 4  & 5  & 6  & 7  & 8 & 9  & 10 & 11 & 12 & 13 & 14 & 15 & 16 & 17 & 18 & 19 & 20 & 21 & 22 & 23 & 24 & 25 & 26 \\ \hline 
      \end{tabular}
    \end{adjustbox}
      \end{table}
  \item Se repite la llave de cifrado las veces que sea necesario para llenar la cadena de texto plano
  \begin{table}[h]
    \begin{adjustbox}{width=1\columnwidth,center}
      \begin{tabular}{|l|l|l|l|l|l|l|l|l|l|l|l|l|l|l|l|l|l|l|l|l|l|l|l|l|l|l|}
        \hline
        Texto   & A  & B  & C & D  & E  & F  & G  & H & I  & J  & K  & L  & M  & N  & O  & P  & Q  & R  & S  & T  & U  & V  & W  & X  & Y  & Z  \\ \hline
        Valor   & 1  & 2  & 3 & 4  & 5  & 6  & 7  & 8 & 9  & 10 & 11 & 12 & 13 & 14 & 15 & 16 & 17 & 18 & 19 & 20 & 21 & 22 & 23 & 24 & 25 & 26 \\ \hline 
        Llave   & L  & L  & A & V  & E  & L  & L  & A & V  & E  & L  & L  & A  & V  & E  & L  & L  & A  & V  & E  & L  & L  & A  & V  & E  & L  \\ \hline
      \end{tabular}
    \end{adjustbox}
  \end{table}
  \newpage
  \item Se suma el valor de la llave con cada carácter del texto plano 
  \begin{table}[h]
    \begin{adjustbox}{width=1\columnwidth,center}
      \begin{tabular}{|l|l|l|l|l|l|l|l|l|l|l|l|l|l|l|l|l|l|l|l|l|l|l|l|l|l|l|}
        \hline
        Texto   & A  & B  & C & D  & E  & F  & G  & H & I  & J  & K  & L  & M  & N  & O  & P  & Q  & R  & S  & T  & U  & V  & W  & X  & Y  & Z  \\ \hline
        Valor   & 1  & 2  & 3 & 4  & 5  & 6  & 7  & 8 & 9  & 10 & 11 & 12 & 13 & 14 & 15 & 16 & 17 & 18 & 19 & 20 & 21 & 22 & 23 & 24 & 25 & 26 \\ \hline 
        Llave   & L  & L  & A & V  & E  & L  & L  & A & V  & E  & L  & L  & A  & V  & E  & L  & L  & A  & V  & E  & L  & L  & A  & V  & E  & L  \\ \hline
        Valor   & 12 & 12 & 1 & 22 & 5  & 12 & 12 & 1 & 22 & 5  & 12 & 12 & 1  & 22 & 5  & 12 & 12 & 1  & 22 & 5  & 12 & 12 & 1  & 22 & 5  & 12 \\ \hline
Suma    & 16 & 14 & 4 & 26 & 10 & 18 & 28 & 9 & 31 & 15 & 23 & 24 & 14 & 36 & 20 & 28 & 29 & 19 & 41 & 25 & 36 & 34 & 24 & 46 & 30 & 38 \\ \hline
      \end{tabular}
    \end{adjustbox}
  \end{table}
  \item Se aplica la función módulo del tamaño del diccionario a cada carácter
  \begin{table}[h]
    \begin{adjustbox}{width=1\columnwidth,center}
      \begin{tabular}{|l|l|l|l|l|l|l|l|l|l|l|l|l|l|l|l|l|l|l|l|l|l|l|l|l|l|l|}
        \hline
        Texto   & A  & B  & C & D  & E  & F  & G  & H & I  & J  & K  & L  & M  & N  & O  & P  & Q  & R  & S  & T  & U  & V  & W  & X  & Y  & Z  \\ \hline
        Valor   & 1  & 2  & 3 & 4  & 5  & 6  & 7  & 8 & 9  & 10 & 11 & 12 & 13 & 14 & 15 & 16 & 17 & 18 & 19 & 20 & 21 & 22 & 23 & 24 & 25 & 26 \\ \hline 
        Llave   & L  & L  & A & V  & E  & L  & L  & A & V  & E  & L  & L  & A  & V  & E  & L  & L  & A  & V  & E  & L  & L  & A  & V  & E  & L  \\ \hline
        Valor   & 12 & 12 & 1 & 22 & 5  & 12 & 12 & 1 & 22 & 5  & 12 & 12 & 1  & 22 & 5  & 12 & 12 & 1  & 22 & 5  & 12 & 12 & 1  & 22 & 5  & 12 \\ \hline
        Suma    & 16 & 14 & 4 & 26 & 10 & 18 & 28 & 9 & 31 & 15 & 23 & 24 & 14 & 36 & 20 & 28 & 29 & 19 & 41 & 25 & 36 & 34 & 24 & 46 & 30 & 38 \\ \hline
        Módulo  & 16 & 14 & 4 & 26 & 10 & 18 & 2  & 9 & 5  & 15 & 23 & 24 & 14 & 10 & 20 & 2  & 3  & 19 & 45 & 25 & 10 & 8  & 24 & 20 & 4  & 12 \\ \hline
      \end{tabular}
    \end{adjustbox}
  \end{table} 
\end{enumerate}
\item Finalmente se convierte el resultado del módulo en su representación de carácter y se concatena para obtener el texto cifrado
\begin{table}[h]
  \begin{adjustbox}{width=1\columnwidth,center}
    \begin{tabular}{|l|l|l|l|l|l|l|l|l|l|l|l|l|l|l|l|l|l|l|l|l|l|l|l|l|l|l|}
      \hline
      Texto   & A  & B  & C & D  & E  & F  & G  & H & I  & J  & K  & L  & M  & N  & O  & P  & Q  & R  & S  & T  & U  & V  & W  & X  & Y  & Z  \\ \hline
      Valor   & 1  & 2  & 3 & 4  & 5  & 6  & 7  & 8 & 9  & 10 & 11 & 12 & 13 & 14 & 15 & 16 & 17 & 18 & 19 & 20 & 21 & 22 & 23 & 24 & 25 & 26 \\ \hline 
      Llave   & L  & L  & A & V  & E  & L  & L  & A & V  & E  & L  & L  & A  & V  & E  & L  & L  & A  & V  & E  & L  & L  & A  & V  & E  & L  \\ \hline
      Valor   & 12 & 12 & 1 & 22 & 5  & 12 & 12 & 1 & 22 & 5  & 12 & 12 & 1  & 22 & 5  & 12 & 12 & 1  & 22 & 5  & 12 & 12 & 1  & 22 & 5  & 12 \\ \hline
      Suma    & 16 & 14 & 4 & 26 & 10 & 18 & 28 & 9 & 31 & 15 & 23 & 24 & 14 & 36 & 20 & 28 & 29 & 19 & 41 & 25 & 36 & 34 & 24 & 46 & 30 & 38 \\ \hline
      Módulo  & 16 & 14 & 4 & 26 & 10 & 18 & 2  & 9 & 5  & 15 & 23 & 24 & 14 & 10 & 20 & 2  & 3  & 19 & 45 & 25 & 10 & 8  & 24 & 20 & 4  & 12 \\ \hline
      Cifrado & P  & N  & D & Z  & J  & R  & B  & I & E  & O  & W  & X  & N  & J  & T  & B  & C  & S  & O  & Y  & J  & H  & X  & T  & D  & L  \\ \hline
    \end{tabular}
  \end{adjustbox}
\end{table} 

\begin{figure}[!ht]
  \centering
  \includegraphics[width=1\textwidth]{histogramaCiphed.png}
  \caption{Histograma comparativo de las frecuencias del idioma, el texto cifrado y el texto plano}
  \label{fig_sim}
\end{figure}
\newpage

\section{Resultados}
\justify
Interfaz de usuario mediante la cual el usuario puede introducir texto desde la interfaz o desde un archivo, ingresar una llave numérica o de texto (según sea el caso) y visualizar el resultado del texto cifrado/descifrado.
Igualmente se permite seleccionar entre los distintos idiomas que se tienen.\\
\begin{figure}[!ht]
  \centering
  \includegraphics[width=0.6\textwidth]{GUI_1}
  \caption{Interfaz de usuario para el cifrado}
  \label{fig_sim}
\end{figure}
\\
Interfaz donde el usuario puede comparar los histogramas de las frecuencias generadas para el idioma, texto original y texto cifrado
\begin{figure}[!ht]
  \centering
  \includegraphics[width=0.6\textwidth]{GUI_2}
  \caption{Histograma comparativo de las frecuencias del idioma, el texto cifrado y el texto plano}
  \label{fig_sim}
\end{figure}
\newpage
\justify
Interfaz donde se permite al usuario agregar/eliminar y modificar el conjunto de caracteres para cada idioma y la frecuencia de sus caracteres así como agregar/eliminar nuevos idiomas.\\
\begin{figure}[!ht]
  \centering
  \includegraphics[width=0.6\textwidth]{GUI_3}
  \caption{Interfaz para administrar los idiomas}
  \label{fig_sim}
\end{figure}
\\
Vista individual del histograma de cada idioma
\begin{figure}[!ht]
  \centering
  \includegraphics[width=0.6\textwidth]{GUI_4}
  \caption{Histograma del idioma}
  \label{fig_sim}
\end{figure}

\newpage

\subsection{Comparativa de histogramas con distintas claves}


\textbf{ABCDEFGHIJKLMNÑOPRSTUVWXYZ}
\begin{figure}[!ht]
  \centering
  \includegraphics[width=1\textwidth]{VigenereAlphabet.jpg}
  \caption{Cifrado con la llave ABCDEFGHIJKLMNÑOPRSTUVWXYZ}
  \label{fig_sim}
\end{figure}

\textbf{THEQUICKBROWNFOXJUMPSOVERTHELAZYDOG}

\begin{figure}[!ht]
  \centering
  \includegraphics[width=1\textwidth]{Vigenere_THEQUICKBROWNFOXJUMPSOVERTHELAZYDOG.jpg}
  \caption{Cifrado con la llave THEQUICKBROWNFOXJUMPSOVERTHELAZYDOG}
  \label{fig_sim}
\end{figure}

\par\vspace{\baselineskip}
%%%%%%%%%%%%%%%%%%%%%%%%%%%%%%%%%%%%%%%%%%%%%%%%%%%%%%%%%%%%%%%%%%%%%%%%%%%%%%%%%%%%%%%
\newpage

\section{Conclusión}
\justify
El método de de Vigenère resulta un tanto mas complejo que el cifrado cesar pero en contraste cuando se utiliza con una llave rica en caracteres resulta bastante más efectivo al momento de generar entropía en el texto cifrado y mitigar ataques de análisis de frecuencia.
Asimismo, se deberían plantear cuestiones que resulten, al menos, interesantes para
realizar como trabajos futuros.

%%%%%%% Bibliografía %%%%%%%%
\bibliographystyle{IEEEtran} 
\addcontentsline{toc}{section}{Referencias}  
\bibliography{IEEEabrv,IEEEreferencias} 
%%%%%%% Bibliografía %%%%%%%%   

\par\vspace{\baselineskip}


\end{document}