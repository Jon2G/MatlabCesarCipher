\documentclass[10pt]{article}

%%%%%%%% PREÁMBULO %%%%%%%%%%%%
\usepackage[spanish,es-tabla]{babel}
\usepackage[utf8]{inputenc} 
\usepackage{multirow} % para las tablas
\usepackage{makecell}
\usepackage{tcolorbox}
\tcbuselibrary{listingsutf8}
\usepackage[spanish]{babel}
\usepackage{fancybox}
\usepackage{courier}
\usepackage{ragged2e}
\usepackage[spanish]{babel}
\usepackage[utf8]{inputenc}
\usepackage{amssymb, amsmath, amsbsy} % simbolitos
\usepackage{upgreek} % para poner letras griegas sin cursiva
\usepackage{cancel} % para tachar
\usepackage{mathdots} % para el comando \iddots
\usepackage{mathrsfs} % para formato de letra
\usepackage{stackrel} % para el comando \stackbin
\usepackage{courier}
\usepackage{subfig}
\usepackage{pdflscape}
\title{Plantilla Tesis SEPI ESIME}
\usepackage[spanish]{babel} %Indica que escribiermos en español
\usepackage[utf8]{inputenc} %Indica qué codificación se está usando ISO-8859-1(latin1)  o utf8  
\usepackage{amsmath} % Comandos extras para matemáticas (cajas para ecuaciones, etc)
\usepackage{amssymb} % Simbolos matematicos (por lo tanto)
\usepackage{graphicx} % Incluir imágenes en LaTeX
\graphicspath{{./../imgs/}}
\usepackage{color} % Para colorear texto
%\usepackage{subfigure} % subfiguras
\usepackage{float} %Podemos usar el especificador [H] en las figuras para que se queden donde queramos
\usepackage{capt-of} % Permite usar etiquetas fuera de elementos flotantes (etiquetas de figuras)
\usepackage{sidecap} % Para poner el texto de las imágenes al lado
\sidecaptionvpos{figure}{c} % Para que el texto se alinie al centro vertical
\usepackage{caption} % Para poder quitar numeracion de figuras
\usepackage{commath} % funcionalidades extras para diferenciales, integrales, etc (\od, \dif, etc)
\usepackage{cancel} % para cancelar expresiones (\cancelto{0}{x})
\usepackage[export]{adjustbox}
\usepackage{anysize} % Para personalizar el ancho de  los márgenes
\marginsize{2cm}{2cm}{2cm}{2cm} % Izquierda, derecha, arriba, abajo

\usepackage{appendix} 
\renewcommand{\appendixname}{Apéndices}
\renewcommand{\appendixtocname}{Apéndices}
\renewcommand{\appendixpagename}{Apéndices} 

% Para que las referencias sean hipervínculos a las figuras o ecuaciones y aparezcan en color
\usepackage[colorlinks=true,plainpages=true,citecolor=blue,linkcolor=blue]{hyperref}
%\usepackage{hyperref}
%Para agregar encabezado y pie de página
\usepackage{fancyhdr} 
\pagestyle{fancy}
\fancyhf{}
\fancyhead[L]{\footnotesize ESIME Culhuacan} %encabezado izquierda
\fancyhead[R]{\footnotesize IPN}   % dereecha
\fancyfoot[R]{\footnotesize E.S.I.M.E }  % Pie derecha
\fancyfoot[C]{\thepage}  % centro
\fancyfoot[L]{\footnotesize MISTI}  %izquierda
\renewcommand{\footrulewidth}{0.4pt}

\usepackage{listings} % Para usar código fuente
\definecolor{dkgreen}{rgb}{0,0.6,0} % Definimos colores para usar en el código
\definecolor{gray}{rgb}{0.5,0.5,0.5} 

% configuración para el lenguaje que queramos utilizar
\usepackage[framed,numbered,autolinebreaks,useliterate]{./../mcode/mcode}

\newcommand{\sen}{\operatorname{\sen}}	% Definimos el comando \sen para el seno en español


\usepackage{longtable}
%%%%%%%% TERMINA PREÁMBULO %%%%%%%%%%%%

\begin{document}
\nocite{IEEEreferencias:Ref1}
\nocite{IEEEreferencias:Ref2}
\nocite{IEEEreferencias:Ref3}
\nocite{IEEEreferencias:Ref4}
\nocite{IEEEreferencias:Ref5}
\nocite{IEEEreferencias:Ref6}
\nocite{IEEEreferencias:Ref7}

\begin{center}											            							%%%
    \newcommand{\HRule}{\rule{\linewidth}{0.5mm}}	                                               %%%\left
    %%%
    \noindent
\makebox[0pt][l]{
    \begin{minipage}{0.49\textwidth} \begin{flushleft}
        \includegraphics[scale = 0.12]{logoesime.png}
        \end{flushleft}\end{minipage}
}\hfill\makebox[0pt]
{
    \begin{minipage}{0.49\textwidth} \begin{center}
        \includegraphics[scale = 0.15]{misti.png}
        \end{center}\end{minipage}
}\hfill%
\makebox[0pt][r]{
    \begin{minipage}{0.49\textwidth} \begin{flushright}
        \includegraphics[scale = 0.25]{IPN.png}
        \end{flushright}\end{minipage}
}\\
\vspace*{1cm}			%%%
    
    \textsc{\huge Instituto Polit\'ecnico Nacional}\\[1cm] 
    \textsc{\LARGE Escuela Superior de Ingenier\'ia Mecanica y Electrica}\\[0.5cm] %%%
    \textsc{\LARGE Unidad Culhuacan}\\[0.5cm] %%%
    \textsc{\LARGE Maestría en Ingeniería en Seguridad y Tecnologías de la información }\\[1cm] %%%
    
    \begin{minipage}{0.9\textwidth} 
    \begin{center}																					%%%
    \textsc{\LARGE REPORTE TÉCNICO}
    \end{center}
    \end{minipage}\\[0.5cm]
                                                                                        %%%
                                                                                        %%%
                 \vspace*{1cm}															%%%
                                                                                        %%%
    \HRule \\[0.1cm]																	%%%
    \begin{center} \textsc{\Large Implementación del Cifrado de Vigenére en Matlab\\}
    \end{center}
    \HRule \\[0.1cm]%%%
                                             %%%
    
                                                                                         %%%
    
    
    
       \vspace{0.8cm}
    \begin{center}
    {\large Presenta}\\                                                                %%%
    Jonathan Eduardo García García\hspace{1cm} \href{mailto:jgarciag1404@alumno.ipn.mx}{jgarciag1404@alumno.ipn.mx}
    \vspace{1 cm}
    \end{center}
    
    \begin{center}
    \begin{minipage}{1\textwidth}													    %%%
    \begin{flushleft} \large														    %%%
    \emph{Profesor:}\\
    \vspace{0.3cm}
    Dr. José Portillo Portillo\\
        Introducción a los sistemas de comunicación seguros.
        \vspace*{1cm}	
        %%%
    \end{flushleft}	
    %%%
    \end{minipage}		                                            %%%
    \end{center}
    
    %\begin{minipage}{1\textwidth}													    %%%
    %\begin{flushleft} \large														    %%%
    %\emph{Directores del proyecto:}\\
    %\vspace{0.3cm}
    %M. en C. Jose Luis Cano Rosas\\
    %Dr. Pedro Guevara L\'opez\
    
                                                                    %%%
     %   \vspace*{1cm}	
                                                                    %%%
    %\end{flushleft}													%%%
    %\end{minipage}		                                            %%%
                                                                    %%%
    %\begin{flushleft}
         
    %\end{flushleft}
    %%%
             
                         %%%
    \vspace{2cm} 																				
    \begin{center}
    {\large \today}													%%%
    \end{center}												  						
    
    \end{center}
\cleardoublepage


\newpage																		
\tableofcontents 





%%%%%%%%%%%%%%%%%%%%%%%%%%%%%%%%%%%%%%%%%%%%%%%%%%%%%%%%%%%%%%%%%%%%%%%%%%%%%%%%%%%%%%%
%RESUMEN
\newpage
\section{Objetivo.}

\begin{enumerate}
  \item Implementar el cifrado Play Fair en matlab
  \item Crear una interfaz de usuario para hacer uso del cifrado
  \item Generar los histogramas del texto cifrado, texto plano y del idioma
\end{enumerate}

  \begin{center}
    \begin{tabular}{ | l | l |}
      \hline
      \thead{\textbf{Equipo necesario}} & \thead{\textbf{Material necesario}}  \\
      \hline
      \makecell[l]{Computadora con el Software Matlab.}&  
			\makecell[l]{Apuntes y conocimientos teóricos sobre el cifrado de Vigenère}  \\
      \hline
    \end{tabular}
  \end{center}

\section{Marco teórico}\index{Introducción o Marco teórico}
\subsection{Playfair y  de Wheatstone}
\justify

Este método lleva por nombre Playfair ya que fue el que se lo presentó a las autoridades inglesas, aunque el verdadero inventor del método fue su amigo el físico Charles Wheatstone, que lo creó para realizar comunicaciones
telegráficas secretas en el año 1854, y se utilizó durante la Primera Guerra Mundial.
Charles Wheatstone fue un inventor y científico inglés del siglo XIX muy conocido por el aparato eléctrico que
lleva su nombre, el puente de Wheatstone, que sirve para medir la resistencia eléctrica.
El método usa una matriz alfabética 5¥5 donde se incluyen las letras del alfabeto inglés (25 de las 26 letras).
En castellano el alfabeto tiene 27 letras, por tanto adaptaremos dicha tabla haciendo que las letras N y Ñ, y las
letras V y W ocupen la misma celda dentro de la tabla.

\begin{figure}[h]
  \centering
  \subfloat[\centering Charles Wheatstone]{{\includegraphics[width=5cm]{charles.jpg} }}
  \qquad
  \subfloat[\centering Lyon Playfair]{{\includegraphics[width=5cm]{playFair.jpg} }}
  \caption{Los amigos Wheatstone y Playfair}
  \label{fig:example}%
\end{figure}
\newpage

\subsection{Cifrado}
El método de cifrado de Playfair es un método de sustitución que cifra pares de caracteres o letras, mediante
una tabla
\begin{figure}[h]
  \centering
  \includegraphics[height=3cm]{playfairKey}
\end{figure}

El método de cifrado trabaja con dos caracteres
(bigrama) a la vez, por lo que el texto en claro se
debe descomponer en parejas de dos caracteres.\\
Cada una de las parejas de caracteres obtenidas
después de la descomposición se sustituye por otra
conforme a las siguientes reglas:\\
Para efectuar el cifrado se siguen las siguientes
reglas:
\begin{enumerate}
  \item Si el par de letras a cifrar están situadas en
  filas y columnas diferentes, se forma el
  rectángulo que tiene como vértices
  opuestos las dos letras. Las letras de los
  otros dos vértices forman el texto cifrado,
  ordenadas por filas de la misma forma que
  en el texto claro. Es decir que se pone
  antes en el texto cifrado la letra que se
  encuentra en la misma fila que la primera
  letra del texto claro. Por ejemplo, con la
  tabla anterior al par li le corresponde FA y
  al par zo le corresponde YB.
  \item Si ambas letras se encuentran en la misma
  fila, se sustituyen por las que se 
  encuentran en la misma fila a su derecha.
  Si alguna de ellas está en la última
  columna se sustituye por la letra de la
  misma fila en la primera columna. Por
  ejemplo ic se cifraría como TA y od como
  BN
  \item Igualmente, si están en la misma columna,
  se cifran mediante las letras que se
  encuentran justamente debajo de ellas. Si
  alguna está en la quinta fila, por la de la
  primera.  
\end{enumerate}

No se pueden cifrar pares compuestos por
letras iguales. La solución es procurar que no
suceda esto por ejemplo introduciendo una
letra con valor nulo entre los dos iguales. Si el
número de letras a cifrar es impar, se le añade
un carácter de relleno

\subsection{Entropia}
\justify
La entropía es un concepto valioso cuando se piensa en hacer 
criptoanálisis dado que representa la medida promedio de 
información que tiene un símbolo en algún mensaje, de hecho se 
puede pensar en calcular la entropía para cierto lenguaje 
(español, inglés, etc.) y es curioso saber que la entropía de cada 
lenguaje tiende a cierto valor característico.
La cantidad de información de un símbolo B se define como:\\
$I(B)=\log_2{\frac{1}{P(B)}}$
\newpage
\section{Desarrollo}
\justify

Se tienen los siguientes alfabetos con su respectiva frecuencia:\\
\textbf{Español:}
\begin{table}[h]
    \begin{adjustbox}{width=\columnwidth,center}
    \begin{tabular}{|l|l|l|l|l|l|l|l|l|l|l|l|l|l|l|l|l|l|l|l|l|l|l|l|l|l|l|}
    \hline
    A & B & C & D & E & F & G & H & I & J  & K  & L  & M  & N  & Ñ  & O  & P  & Q  & R  & S  & T  & U  & V  & W  & X  & Y  & Z  \\ \hline
    1 & 2 & 3 & 4 & 5 & 6 & 7 & 8 & 9 & 10 & 11 & 12 & 13 & 14 & 15 & 16 & 17 & 18 & 19 & 20 & 21 & 22 & 23 & 24 & 25 & 26 & 27\\ \hline
    \%12.53&\%1.42&\%4.68&\%5.86&\%13.68&\%0.69&\%1.01&\%0.70&\%6.25&\%0.44&\%0.02&\%4.97&\%3.15&\%6.71&\%0.31&\%8.68&\%2.51&\%0.88&\%6.87&\%7.98&\%4.63&\%3.93&\%0.90&\%0.01&\%0.22&\%0.90&\%0.52 \\ \hline
    \end{tabular}
\end{adjustbox}
    \end{table}

    \begin{figure}[!ht]
      \centering
      \includegraphics[width=0.7\textwidth]{histogramaSpanish.png}
      \caption{Histograma de frecuencias del idioma Español}
      \label{fig_sim}
    \end{figure}

    \textbf{Inglés:}
    \begin{table}[h]
        \begin{adjustbox}{width=\columnwidth,center}
        \begin{tabular}{|l|l|l|l|l|l|l|l|l|l|l|l|l|l|l|l|l|l|l|l|l|l|l|l|l|l|l|}
        \hline
        A & B & C & D & E & F & G & H & I & J  & K  & L  & M  & N  & O  & P  & Q  & R  & S  & T  & U  & V  & W  & X  & Y  & Z  \\ \hline
        1 & 2 & 3 & 4 & 5 & 6 & 7 & 8 & 9 & 10 & 11 & 12 & 13 & 14 & 15 & 16 & 17 & 18 & 19 & 20 & 21 & 22 & 23 & 24 & 25 & 26 \\ \hline
        \%8.34&\%1.54&\%2.73&\%4.14&\%12.6&\%2.03&\%1.92&\%6.11&\%6.71&\%0.23&\%0.87&\%4.24&\%2.53&\%6.80&\%7.70&\%1.66&\%0.09&\%5.68&\%6.11&\%9.37&\%2.85&\%1.06&\%2.34&\%0.20&\%2.04&\%0.06 \\ \hline
        \end{tabular}
    \end{adjustbox}
        \end{table}
        \begin{figure}[!ht]
          \centering
          \includegraphics[width=0.7\textwidth]{histogramaSpanish.png}
          \caption{Histograma de frecuencias del idioma Inglés}
          \label{fig_sim}
        \end{figure}
\newpage
  \subsection{Proceso de expansión de la llave de cifrado}
  \lstinputlisting{./../code/keyPlayFair.m}

    Se toman los caracteres de la llave y se busca su representación numérica en el alfabeto:
    \begin{table}[h!]
      \begin{adjustbox}{width=0.3\columnwidth,center}
        \begin{tabular}{|l|l|l|l|l|l|}
      \hline
      Llave & L  & L  & A & V  & E \\ \hline
      Valor & 12 & 12 & 1 & 22 & 5 \\ \hline
    \end{tabular}
  \end{adjustbox}
\end{table}
\\
Estos valores servirán como alfabeto para intercambiar los caracteres
\begin{center}
  \begin{table}[ht]
    \begin{adjustbox}{width=0.4\columnwidth,center}
    \begin{tabular}{|l|l|l|l|l|}
    \hline
    P & L & A & Y & F \\ \hline
    I/J & R & B & C & D \\ \hline
    E & F & G & H & K \\ \hline
    M & N & O & Q & S \\ \hline
    T & U & V & W & X/Z \\ \hline
    \end{tabular}
  \end{adjustbox}
    \end{table}
\end{center}




\newpage


\subsection{Implementación del cifrado en matlab}      
\lstinputlisting{./../code/playFair.m}
\newpage

\begin{figure}[!ht]
  \centering
  \includegraphics[width=1\textwidth]{histogramaCiphedPlayFair.png}
  \caption{Histograma comparativo de las frecuencias del idioma, el texto cifrado y el texto plano}
  \label{fig_sim}
\end{figure}

\section{Resultados}
\justify
Interfaz de usuario mediante la cual el usuario puede introducir texto desde la interfaz o desde un archivo, ingresar una llave numérica o de texto (según sea el caso) y visualizar el resultado del texto cifrado/descifrado.
Igualmente se permite seleccionar entre los distintos idiomas que se tienen.\\
\begin{figure}[!ht]
  \centering
  \includegraphics[width=0.5\textwidth]{GUI_1PF}
  \caption{Interfaz de usuario para el cifrado}
  \label{fig_sim}
\end{figure}
\\
Interfaz donde el usuario puede comparar los histogramas de las frecuencias generadas para el idioma, texto original y texto cifrado
\begin{figure}[!ht]
  \centering
  \includegraphics[width=0.6\textwidth]{GUI_2PF}
  \caption{Histograma comparativo de las frecuencias del idioma, el texto cifrado y el texto plano}
  \label{fig_sim}
\end{figure}
\newpage
\justify
Interfaz donde se permite al usuario agregar/eliminar y modificar el conjunto de caracteres para cada idioma y la frecuencia de sus caracteres así como agregar/eliminar nuevos idiomas.\\
\begin{figure}[!ht]
  \centering
  \includegraphics[width=0.6\textwidth]{GUI_3}
  \caption{Interfaz para administrar los idiomas}
  \label{fig_sim}
\end{figure}
\\
Vista individual del histograma de cada idioma
\begin{figure}[!ht]
  \centering
  \includegraphics[width=0.6\textwidth]{GUI_4}
  \caption{Histograma del idioma}
  \label{fig_sim}
\end{figure}

\newpage

\subsection{Comparativa de histogramas con distintas claves}


\textbf{ABCDEFGHIJKLMNÑOPRSTUVWXYZ}
\begin{figure}[!ht]
  \centering
  \includegraphics[width=1\textwidth]{PlayFairAlphabet}
  \caption{Cifrado con la llave ABCDEFGHIJKLMNÑOPRSTUVWXYZ}
  \label{fig_sim}
\end{figure}
\newpage
\textbf{THEQUICKBROWNFOXJUMPSOVERTHELAZYDOG}

\begin{figure}[!ht]
  \centering
  \includegraphics[width=1\textwidth]{PlayFair_THEQUICKBROWNFOXJUMPSOVERTHELAZYDOG}
  \caption{Cifrado con la llave THEQUICKBROWNFOXJUMPSOVERTHELAZYDOG}
  \label{fig_sim}
\end{figure}

\par\vspace{\baselineskip}
%%%%%%%%%%%%%%%%%%%%%%%%%%%%%%%%%%%%%%%%%%%%%%%%%%%%%%%%%%%%%%%%%%%%%%%%%%%%%%%%%%%%%%%

\section{Conclusión}
\justify
Este método es también de sustitución al igual que el cifrado de César, lo único que en el método de Playfair se
usan pares de letras o caracteres, en vez de ir de letra en letra.\\
Asimismo, se deberían plantear cuestiones que resulten, al menos, interesantes para
realizar como trabajos futuros.

%%%%%%% Bibliografía %%%%%%%%
\bibliographystyle{IEEEtran} 
\addcontentsline{toc}{section}{Referencias}  
\bibliography{IEEEabrv,IEEEreferencias} 
%%%%%%% Bibliografía %%%%%%%%   

\par\vspace{\baselineskip}


\end{document}