\documentclass[10pt]{article}

%%%%%%%% PREÁMBULO %%%%%%%%%%%%
\usepackage[spanish,es-tabla]{babel}
\usepackage[utf8]{inputenc} 
\usepackage{multirow} % para las tablas
\usepackage{makecell}
\usepackage{tcolorbox}
\tcbuselibrary{listingsutf8}
\usepackage[spanish]{babel}
\usepackage{fancybox}
\usepackage{courier}
\usepackage{ragged2e}
\usepackage[spanish]{babel}
\usepackage[utf8]{inputenc}
\usepackage{amssymb, amsmath, amsbsy} % simbolitos
\usepackage{upgreek} % para poner letras griegas sin cursiva
\usepackage{cancel} % para tachar
\usepackage{mathdots} % para el comando \iddots
\usepackage{mathrsfs} % para formato de letra
\usepackage{stackrel} % para el comando \stackbin
\usepackage{courier}
\usepackage{subfig}
\usepackage{pdflscape}
\title{Plantilla Tesis SEPI ESIME}
\usepackage[spanish]{babel} %Indica que escribiermos en español
\usepackage[utf8]{inputenc} %Indica qué codificación se está usando ISO-8859-1(latin1)  o utf8  
\usepackage{amsmath} % Comandos extras para matemáticas (cajas para ecuaciones, etc)
\usepackage{amssymb} % Simbolos matematicos (por lo tanto)
\usepackage{graphicx} % Incluir imágenes en LaTeX
\graphicspath{{./../imgs/}}
\usepackage{color} % Para colorear texto
%\usepackage{subfigure} % subfiguras
\usepackage{float} %Podemos usar el especificador [H] en las figuras para que se queden donde queramos
\usepackage{capt-of} % Permite usar etiquetas fuera de elementos flotantes (etiquetas de figuras)
\usepackage{sidecap} % Para poner el texto de las imágenes al lado
\sidecaptionvpos{figure}{c} % Para que el texto se alinie al centro vertical
\usepackage{caption} % Para poder quitar numeracion de figuras
\usepackage{commath} % funcionalidades extras para diferenciales, integrales, etc (\od, \dif, etc)
\usepackage{cancel} % para cancelar expresiones (\cancelto{0}{x})
\usepackage[export]{adjustbox}
\usepackage{anysize} % Para personalizar el ancho de  los márgenes
\marginsize{2cm}{2cm}{2cm}{2cm} % Izquierda, derecha, arriba, abajo

\usepackage{appendix} 
\renewcommand{\appendixname}{Apéndices}
\renewcommand{\appendixtocname}{Apéndices}
\renewcommand{\appendixpagename}{Apéndices} 

% Para que las referencias sean hipervínculos a las figuras o ecuaciones y aparezcan en color
\usepackage[colorlinks=true,plainpages=true,citecolor=blue,linkcolor=blue]{hyperref}
%\usepackage{hyperref}
%Para agregar encabezado y pie de página
\usepackage{fancyhdr} 
\pagestyle{fancy}
\fancyhf{}
\fancyhead[L]{\footnotesize ESIME Culhuacan} %encabezado izquierda
\fancyhead[R]{\footnotesize IPN}   % dereecha
\fancyfoot[R]{\footnotesize E.S.I.M.E }  % Pie derecha
\fancyfoot[C]{\thepage}  % centro
\fancyfoot[L]{\footnotesize MISTI}  %izquierda
\renewcommand{\footrulewidth}{0.4pt}

\usepackage{listings} % Para usar código fuente
\definecolor{dkgreen}{rgb}{0,0.6,0} % Definimos colores para usar en el código
\definecolor{gray}{rgb}{0.5,0.5,0.5} 

% configuración para el lenguaje que queramos utilizar
\usepackage[framed,numbered,autolinebreaks,useliterate]{./../mcode/mcode}

\newcommand{\sen}{\operatorname{\sen}}	% Definimos el comando \sen para el seno en español


\usepackage{longtable}
%%%%%%%% TERMINA PREÁMBULO %%%%%%%%%%%%

\begin{document}
\nocite{IEEEreferencias:Ref1}
\nocite{IEEEreferencias:Ref2}
\nocite{IEEEreferencias:Ref3}
\nocite{IEEEreferencias:Ref4}
\nocite{IEEEreferencias:Ref5}
\nocite{IEEEreferencias:Ref6}
\nocite{IEEEreferencias:Ref7}

\begin{center}											            							%%%
    \newcommand{\HRule}{\rule{\linewidth}{0.5mm}}	                                               %%%\left
    %%%
    \noindent
\makebox[0pt][l]{
    \begin{minipage}{0.49\textwidth} \begin{flushleft}
        \includegraphics[scale = 0.12]{logoesime.png}
        \end{flushleft}\end{minipage}
}\hfill\makebox[0pt]
{
    \begin{minipage}{0.49\textwidth} \begin{center}
        \includegraphics[scale = 0.15]{misti.png}
        \end{center}\end{minipage}
}\hfill%
\makebox[0pt][r]{
    \begin{minipage}{0.49\textwidth} \begin{flushright}
        \includegraphics[scale = 0.25]{IPN.png}
        \end{flushright}\end{minipage}
}\\
\vspace*{1cm}			%%%
    
    \textsc{\huge Instituto Polit\'ecnico Nacional}\\[1cm] 
    \textsc{\LARGE Escuela Superior de Ingenier\'ia Mecanica y Electrica}\\[0.5cm] %%%
    \textsc{\LARGE Unidad Culhuacan}\\[0.5cm] %%%
    \textsc{\LARGE Maestría en Ingeniería en Seguridad y Tecnologías de la información }\\[1cm] %%%
    
    \begin{minipage}{0.9\textwidth} 
    \begin{center}																					%%%
    \textsc{\LARGE REPORTE TÉCNICO}
    \end{center}
    \end{minipage}\\[0.5cm]
                                                                                        %%%
                                                                                        %%%
                 \vspace*{1cm}															%%%
                                                                                        %%%
    \HRule \\[0.1cm]																	%%%
    \begin{center} \textsc{\Large Implementación del Cifrado de Vigenére en Matlab\\}
    \end{center}
    \HRule \\[0.1cm]%%%
                                             %%%
    
                                                                                         %%%
    
    
    
       \vspace{0.8cm}
    \begin{center}
    {\large Presenta}\\                                                                %%%
    Jonathan Eduardo García García\hspace{1cm} \href{mailto:jgarciag1404@alumno.ipn.mx}{jgarciag1404@alumno.ipn.mx}
    \vspace{1 cm}
    \end{center}
    
    \begin{center}
    \begin{minipage}{1\textwidth}													    %%%
    \begin{flushleft} \large														    %%%
    \emph{Profesor:}\\
    \vspace{0.3cm}
    Dr. José Portillo Portillo\\
        Introducción a los sistemas de comunicación seguros.
        \vspace*{1cm}	
        %%%
    \end{flushleft}	
    %%%
    \end{minipage}		                                            %%%
    \end{center}
    
    %\begin{minipage}{1\textwidth}													    %%%
    %\begin{flushleft} \large														    %%%
    %\emph{Directores del proyecto:}\\
    %\vspace{0.3cm}
    %M. en C. Jose Luis Cano Rosas\\
    %Dr. Pedro Guevara L\'opez\
    
                                                                    %%%
     %   \vspace*{1cm}	
                                                                    %%%
    %\end{flushleft}													%%%
    %\end{minipage}		                                            %%%
                                                                    %%%
    %\begin{flushleft}
         
    %\end{flushleft}
    %%%
             
                         %%%
    \vspace{2cm} 																				
    \begin{center}
    {\large \today}													%%%
    \end{center}												  						
    
    \end{center}
\cleardoublepage


\newpage																		
\tableofcontents 





%%%%%%%%%%%%%%%%%%%%%%%%%%%%%%%%%%%%%%%%%%%%%%%%%%%%%%%%%%%%%%%%%%%%%%%%%%%%%%%%%%%%%%%
%RESUMEN
\newpage
\section{Objetivo.}

\begin{enumerate}
  \item Implementar el cifrado Hill en matlab
  \item Crear una interfaz de usuario para hacer uso del cifrado
  \item Generar los histogramas del texto cifrado, texto plano y del idioma
\end{enumerate}

  \begin{center}
    \begin{tabular}{ | l | l |}
      \hline
      \thead{\textbf{Equipo necesario}} & \thead{\textbf{Material necesario}}  \\
      \hline
      \makecell[l]{Computadora con el Software Matlab.}&  
			\makecell[l]{Apuntes y conocimientos teóricos sobre el cifrado de Vigenère}  \\
      \hline
    \end{tabular}
  \end{center}

\section{Marco teórico}\index{Introducción o Marco teórico}
\subsection{Cifrado Hill}
\justify

Lester Hill propuso en 1929 su método de cifrado de
un mensaje en la revista The American Mathematical
Monthly. La idea es bastante sencilla, como vamos
a exponer a continuación directamente adaptada al
caso de imágenes digitales y con un ejemplo concreto
que siempre se entiende mejor.
Una imagen digital no es más que una matriz de
números. Por ejemplo, la imagen de la figura 1 es una
matriz de tamaño
256 donde los niveles de gris
de la imagen varían desde 0 correspondientes al negro
hasta 255 correspondientes al blanco. Los números
comprendidos entre 0 y 255 se escriben en binario con
8 bits, por eso esta imagen necesita 1 byte por píxel.
Usaremos una matriz secreta K solo conocida por
emisor y receptor

\subsection{Cifrado}

\begin{enumerate}
  \item Cifra el texto en claro considerando bloques de k símbolos del mismo (bigramas, trigramas, etc.).
  \item Propone un criptosistema utilizando como clave (K) una matriz cuadrada de orden k, es decir, una matriz con el mismo número de filas (k) y de columnas (k), de tal forma que los caracteres de la clave se disponen en filas y columnas (de izquierda a derecha y de arriba a abajo), y el algoritmo de cifrado para el primer bloque de k símbolos del texto en claro sería el siguiente:
    
  \item Igualmente, si están en la misma columna,
  se cifran mediante las letras que se
  encuentran justamente debajo de ellas. Si
  alguna está en la quinta fila, por la de la
  primera.  
  \begin{figure}[!ht]
    \centering
    \includegraphics[width=0.7\textwidth]{Criptografía_Cifrado de Hill.png}
    \label{fig_sim}
  \end{figure}
  Donde:\\
kij: carácter de la fila i columna j de la matriz (K) que se utilizará como clave para cifrar el texto en claro.\\
mi: carácter i-ésimo del mensaje a cifrar o texto en claro.\\
ci: carácter i-ésimo del mensaje cifrado o criptograma.\\
n: tamaño del alfabeto.
  \item Conforme a este criptosistema, la matriz K debe tener inversa K-1 mod n, que será la que se use en el descifrado del criptograma.
\end{enumerate}
\newpage
\subsection{Entropia}
\justify
La entropía es un concepto valioso cuando se piensa en hacer 
criptoanálisis dado que representa la medida promedio de 
información que tiene un símbolo en algún mensaje, de hecho se 
puede pensar en calcular la entropía para cierto lenguaje 
(español, inglés, etc.) y es curioso saber que la entropía de cada 
lenguaje tiende a cierto valor característico.
La cantidad de información de un símbolo B se define como:\\
$I(B)=\log_2{\frac{1}{P(B)}}$

\section{Desarrollo}
\justify

Se tienen los siguientes alfabetos con su respectiva frecuencia:\\
\textbf{Español:}
\begin{table}[h]
    \begin{adjustbox}{width=\columnwidth,center}
    \begin{tabular}{|l|l|l|l|l|l|l|l|l|l|l|l|l|l|l|l|l|l|l|l|l|l|l|l|l|l|l|}
    \hline
    A & B & C & D & E & F & G & H & I & J  & K  & L  & M  & N  & Ñ  & O  & P  & Q  & R  & S  & T  & U  & V  & W  & X  & Y  & Z  \\ \hline
    1 & 2 & 3 & 4 & 5 & 6 & 7 & 8 & 9 & 10 & 11 & 12 & 13 & 14 & 15 & 16 & 17 & 18 & 19 & 20 & 21 & 22 & 23 & 24 & 25 & 26 & 27\\ \hline
    \%12.53&\%1.42&\%4.68&\%5.86&\%13.68&\%0.69&\%1.01&\%0.70&\%6.25&\%0.44&\%0.02&\%4.97&\%3.15&\%6.71&\%0.31&\%8.68&\%2.51&\%0.88&\%6.87&\%7.98&\%4.63&\%3.93&\%0.90&\%0.01&\%0.22&\%0.90&\%0.52 \\ \hline
    \end{tabular}
\end{adjustbox}
    \end{table}

    \begin{figure}[!ht]
      \centering
      \includegraphics[width=0.6\textwidth]{histogramaSpanish.png}
      \caption{Histograma de frecuencias del idioma Español}
      \label{fig_sim}
    \end{figure}

    \textbf{Inglés:}
    \begin{table}[h]
        \begin{adjustbox}{width=\columnwidth,center}
        \begin{tabular}{|l|l|l|l|l|l|l|l|l|l|l|l|l|l|l|l|l|l|l|l|l|l|l|l|l|l|l|}
        \hline
        A & B & C & D & E & F & G & H & I & J  & K  & L  & M  & N  & O  & P  & Q  & R  & S  & T  & U  & V  & W  & X  & Y  & Z  \\ \hline
        1 & 2 & 3 & 4 & 5 & 6 & 7 & 8 & 9 & 10 & 11 & 12 & 13 & 14 & 15 & 16 & 17 & 18 & 19 & 20 & 21 & 22 & 23 & 24 & 25 & 26 \\ \hline
        \%8.34&\%1.54&\%2.73&\%4.14&\%12.6&\%2.03&\%1.92&\%6.11&\%6.71&\%0.23&\%0.87&\%4.24&\%2.53&\%6.80&\%7.70&\%1.66&\%0.09&\%5.68&\%6.11&\%9.37&\%2.85&\%1.06&\%2.34&\%0.20&\%2.04&\%0.06 \\ \hline
        \end{tabular}
    \end{adjustbox}
        \end{table}
        \begin{figure}[!ht]
          \centering
          \includegraphics[width=0.7\textwidth]{histogramaSpanish.png}
          \caption{Histograma de frecuencias del idioma Inglés}
          \label{fig_sim}
        \end{figure}
\newpage
  \subsection{Generación de llave de cifrado}
  \lstinputlisting{./../code/keyHill.m}

    Se toman los caracteres de la llave y se busca su representación numérica en el alfabeto:
    \begin{table}[h!]
      \begin{adjustbox}{width=0.3\columnwidth,center}
        \begin{tabular}{|l|l|l|l|l|l|}
      \hline
      Llave & L  & L  & A & V  & E \\ \hline
      Valor & 12 & 12 & 1 & 22 & 5 \\ \hline
    \end{tabular}
  \end{adjustbox}
\end{table}
\\
Estos valores servirán como alfabeto para intercambiar los caracteres


\newpage


\subsection{Implementación del cifrado en matlab}      
\lstinputlisting{./../code/hill.m}
\newpage

\begin{figure}[!ht]
  \centering
  \includegraphics[width=1\textwidth]{hillHistogram.jpg}
  \caption{Histograma comparativo de las frecuencias del idioma, el texto cifrado y el texto plano}
  \label{fig_sim}
\end{figure}

\section{Resultados}
\justify
Interfaz de usuario mediante la cual el usuario puede introducir texto desde la interfaz o desde un archivo, ingresar una llave numérica o de texto (según sea el caso) y visualizar el resultado del texto cifrado/descifrado.
Igualmente se permite seleccionar entre los distintos idiomas que se tienen.\\
\begin{figure}[!ht]
  \centering
  \includegraphics[width=0.5\textwidth]{GUI_1H}
  \caption{Interfaz de usuario para el cifrado}
  \label{fig_sim}
\end{figure}
\\
Interfaz donde el usuario puede comparar los histogramas de las frecuencias generadas para el idioma, texto original y texto cifrado
\begin{figure}[!ht]
  \centering
  \includegraphics[width=0.6\textwidth]{GUI_2PF}
  \caption{Histograma comparativo de las frecuencias del idioma, el texto cifrado y el texto plano}
  \label{fig_sim}
\end{figure}
\newpage
\justify
Interfaz donde se permite al usuario agregar/eliminar y modificar el conjunto de caracteres para cada idioma y la frecuencia de sus caracteres así como agregar/eliminar nuevos idiomas.\\
\begin{figure}[!ht]
  \centering
  \includegraphics[width=0.6\textwidth]{GUI_3}
  \caption{Interfaz para administrar los idiomas}
  \label{fig_sim}
\end{figure}
\\
Vista individual del histograma de cada idioma
\begin{figure}[!ht]
  \centering
  \includegraphics[width=0.6\textwidth]{GUI_4}
  \caption{Histograma del idioma}
  \label{fig_sim}
\end{figure}

\newpage

\par\vspace{\baselineskip}
%%%%%%%%%%%%%%%%%%%%%%%%%%%%%%%%%%%%%%%%%%%%%%%%%%%%%%%%%%%%%%%%%%%%%%%%%%%%%%%%%%%%%%%

\section{Conclusión}
\justify
Este método es también de sustitución al igual que el cifrado de César, lo único que en el método de Playfair se
usan pares de letras o caracteres, en vez de ir de letra en letra.\\
Asimismo, se deberían plantear cuestiones que resulten, al menos, interesantes para
realizar como trabajos futuros.

%%%%%%% Bibliografía %%%%%%%%
\bibliographystyle{IEEEtran} 
\addcontentsline{toc}{section}{Referencias}  
\bibliography{IEEEabrv,IEEEreferencias} 
%%%%%%% Bibliografía %%%%%%%%   

\par\vspace{\baselineskip}


\end{document}